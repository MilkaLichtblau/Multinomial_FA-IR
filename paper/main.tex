\documentclass[acmsmall]{acmart}

%%%%%%%%%%%%%%%%%%%%%%%%%%%%%%%%%%%%%%%%%%%%%%%%%%%%%%%%%%%%%%%%%%
% BEFORE SUBMITTING
%%%%%%%%%%%%%%%%%%%%%%%%%%%%%%%%%%%%%%%%%%%%%%%%%%%%%%%%%%%%%%%%%%

% Before submitting for review
% - Spellcheck
% - Check CCS concepts
% - Change [show] to [hide] in "\usepackage[show]{chato-notes}"
\usepackage[show]{chato-notes}  % \note, \todo, \inote, \citemissing
% - Remove next line, which deletes margins for easier visualization
\usepackage[a4,center,noinfo,cross,width=14.5cm,height=21.5cm]{crop}

%%%%%%%%%%%%%%%%%%%%%%%%%%%%%%%%%%%%%%%%%%%%%%%%%%%%%%%%%%%%%%%%%%%%%%%%%%
% PACKAGES
%%%%%%%%%%%%%%%%%%%%%%%%%%%%%%%%%%%%%%%%%%%%%%%%%%%%%%%%%%%%%%%%%%%%%%%%%%
\usepackage{lipsum}
\usepackage{array}
\usepackage{paralist} % compact itemizations and enumerations
\usepackage{subfig}% sub-figures
\usepackage{makecell,amsmath}
\usepackage{forest} %drawing trees
% Algorithms
\usepackage[ruled,lined,linesnumbered]{algorithm2e}
% Hyperlinked references
\usepackage{hyperref}
\usepackage{xcolor}
\usepackage{color, colortbl}
\usepackage{float}
\usepackage{graphics}
\usepackage[section]{placeins}
\hypersetup{
	colorlinks=false,
	linkcolor={red!20!black},
	citecolor={green!20!black},
	urlcolor={blue!20!black}
}

%%
%% \BibTeX command to typeset BibTeX logo in the docs
\AtBeginDocument{%
  \providecommand\BibTeX{{%
    \normalfont B\kern-0.5em{\scshape i\kern-0.25em b}\kern-0.8em\TeX}}}

%% Rights management information.  This information is sent to you
%% when you complete the rights form.  These commands have SAMPLE
%% values in them; it is your responsibility as an author to replace
%% the commands and values with those provided to you when you
%% complete the rights form.
\setcopyright{acmcopyright}
\copyrightyear{2019}
\acmYear{2019}
\acmDOI{tba}


%%
%% These commands are for a JOURNAL article.
%\acmJournal{JACM}
%\acmVolume{37}
%\acmNumber{4}
%\acmArticle{111}
%\acmMonth{8}

%%
%% Submission ID.
%% Use this when submitting an article to a sponsored event. You'll
%% receive a unique submission ID from the organizers
%% of the event, and this ID should be used as the parameter to this command.
%%\acmSubmissionID{123-A56-BU3}

%%
%% The majority of ACM publications use numbered citations and
%% references.  The command \citestyle{authoryear} switches to the
%% "author year" style.
%%
%% If you are preparing content for an event
%% sponsored by ACM SIGGRAPH, you must use the "author year" style of
%% citations and references.
%% Uncommenting
%% the next command will enable that style.
%%\citestyle{acmauthoryear}

%%%%%%%%%%%%%%%%%%%%%%%%%%%%%%%%%%%%%%%%%%%%%%%%%%%%%%%%%%%%%%%%%
% NEW COMMANDS
%%%%%%%%%%%%%%%%%%%%%%%%%%%%%%%%%%%%%%%%%%%%%%%%%%%%%%%%%%%%%%%%%
\newcolumntype{L}[1]{>{\raggedright\let\newline\\\arraybackslash\hspace{0pt}}m{#1}}
\newcolumntype{C}[1]{>{\centering\let\newline\\\arraybackslash\hspace{0pt}}m{#1}}
\newcolumntype{R}[1]{>{\raggedleft\let\newline\\\arraybackslash\hspace{0pt}}m{#1}}
%%%%%%%%%%%%%%%%%%%%%%%%%%%%%%%%%%%%%%%%%%%%%%%%%%%%%%%%%%%%%%%%%%
% ALIASES
%%%%%%%%%%%%%%%%%%%%%%%%%%%%%%%%%%%%%%%%%%%%%%%%%%%%%%%%%%%%%%%%%%
\newcommand{\methodname}{\textsc{FA*IR}\xspace}
\newcommand{\algoFAIR}[0]{{\textsc{FA*IR}}\xspace}
\newcommand{\algoFAIRBF}[0]{{\textsc{\textbf{FA*IR}}}\xspace}
\newcommand{\adj}[0]{\ensuremath{\operatorname{c}}}
\newcommand{\alphaadj}[0]{\ensuremath{\alpha_\textit{corr}}}
\newcommand{\mfail}{\ensuremath{\operatorname{fail}}}
\newcommand{\msucc}{\ensuremath{\operatorname{succ}}}
\newcommand{\minv}{\ensuremath{m^{-1}}}
\newcommand{\mtree}{\ensuremath{\operatorname{mtree}}}
\newcommand{\mtable}{\ensuremath{\operatorname{mTable}}}
\newcommand{\spara}[1]{\smallskip\noindent{\bf #1}}
\newcommand{\failprob}{\ensuremath{P_{\operatorname{fail}}}}
\newcommand{\algoCorrect}[0]{{\sc AdjustSignificance}\xspace}
\newcommand{\algoRecursive}[0]{{\sc SuccessProbability}\xspace}
\newcommand{\algoBinomBinary}[0]{{\sc AlphaAdjustment}\xspace}
\newcommand{\algoComputeMTree}[0]{{\sc ComputeMTree}\xspace}
\newcommand{\algoImcdf}[0]{{\sc InverseMultinomialCDF}\xspace}
\newcommand{\algoReg}[0]{{\sc RegressionAdjustment}\xspace}
\newcommand{\algoMtable}[0]{{\sc ConstructMTable}\xspace}
\newcommand{\algoMultBinary}[0]{{\sc MultinomialAlphaAdjustment}\xspace}
\newcommand\mycommfont[1]{\scriptsize\ttfamily\textcolor{blue}{#1}}
\SetCommentSty{mycommfont}
\SetKwInOut{AlgInput}{input}
\SetKwInOut{AlgOutput}{output}
\SetKwComment{AlgComment}{// }{}
\SetAlCapFnt{\small}
\SetAlCapNameFnt{\footnotesize}
\newcommand{\nosemic}{\renewcommand{\@endalgocfline}{\relax}}% Drop semi-colon ;
\newcommand{\dosemic}{\renewcommand{\@endalgocfline}{\algocf@endline}}% Reinstate semi-colon ;
\newcommand{\pushline}{\Indp}% Indent
\newcommand{\popline}{\Indm}% Undent

\newcommand{\meike}[1]{\textcolor{magenta}{{\bf [MZ: }{\em #1}{\bf ]}}}
\newcommand{\tablemargin}{\vspace{-5mm}}
\newcommand{\tablemargintop}{\vspace{-3mm}}
\newcommand{\CaptionMargin}{\vspace{-3mm}}
%%
%% end of the preamble, start of the body of the document source.
\begin{document}

%%
%% The "title" command has an optional parameter,
%% allowing the author to define a "short title" to be used in page headers.
%%%%%%%%%%%%%%%%%%%%%%%%%%%%%%%%%%%%%%%%%%%%%%%%%%%%%%%%%%%%%%%%%
% TITLE
%%%%%%%%%%%%%%%%%%%%%%%%%%%%%%%%%%%%%%%%%%%%%%%%%%%%%%%%%%%%%%%%%
\title{Group Fairness for Rankings with Multiple Protected Groups using \methodname}
\titlenote{This is an extended version of \citet{zehlike2017fair}, in which \methodname was introduced for a single protected attribute. This extended version includes multiple protected attributes.}

%%%%%%%%%%%%%%%%%%%%%%%%%%%%%%%%%%%%%%%%%%%%%%%%%%%%%%%%%%%%%%%%%
% AUTHORS
%%%%%%%%%%%%%%%%%%%%%%%%%%%%%%%%%%%%%%%%%%%%%%%%%%%%%%%%%%%%%%%%%
\author{Meike Zehlike}
\affiliation{
	\institution{Humboldt Universit\"at zu Berlin}
}
\affiliation{
	\institution{Max-Planck-Inst. for Software Systems}
	\streetaddress{Campus E1 5}
	\postcode{66123}
	\city{Saarbr\"ucken}
	\country{Germany}
}
\email{meikezehlike@mpi-sws.org}

\author{Tom S\"uhr}
\affiliation{
	\institution{Technische Universit\"at Berlin}}
\email{tom.suehr@googlemail.com}

\author{Carlos Castillo}
\affiliation{
	\institution{Universitat Pompeu Fabra}}
\email{chato@acm.org}

%%
%% By default, the full list of authors will be used in the page
%% headers. Often, this list is too long, and will overlap
%% other information printed in the page headers. This command allows
%% the author to define a more concise list
%% of authors' names for this purpose.
\renewcommand{\shortauthors}{Zehlike et al.}

\begin{abstract}
	In this work, we define and solve the Fair Top-$k$ Ranking problem, in which we want to determine a subset of $k$ candidates from a large pool of $n \gg k$ candidates, maximizing utility (i.e., select the ``best'' candidates) subject to group fairness criteria.

	Our ranked group fairness definition extends group fairness using the standard notion of protected groups and is based on ensuring that the proportion of protected candidates in every prefix of the top-$k$ ranking remains statistically above or indistinguishable from a given minimum.
	%
	Utility is operationalized in two ways:
	\begin{inparaenum}[(i)]
		\item every candidate included in the top-$k$ should be more qualified than every candidate not included; and
		\item for every pair of candidates in the top-$k$, the more qualified candidate should be ranked above.
	\end{inparaenum}

	We present an efficient algorithm for producing the Fair Top-$k$ Ranking, and tested experimentally on existing datasets as well as new datasets released with this paper, showing that our approach yields small distortions with respect to rankings that maximize utility without considering fairness criteria.
	%
	%To the best of our knowledge, this is the first algorithm grounded in statistical tests that can mitigate biases in the representation of an under-represented group along a ranked list.

	Unlike the first version of FA*IR described in Zehlike et al. (2017), %write the citation like this, so the abstract is self-contained
  we consider that more than one protected group is present, which means that a statistical test based on a multinomial distribution needs to be used instead of one for a binomial distribution.
	%
	This poses an important technical challenge and increases both the space and time complexity of the re-ranking algorithm.
\end{abstract}


%%
%% The code below is generated by the tool at http://dl.acm.org/ccs.cfm.
%% Please copy and paste the code instead of the example below.
%%
%\begin{CCSXML}
%<ccs2012>
% <concept>
%  <concept_id>10010520.10010553.10010562</concept_id>
%  <concept_desc>Computer systems organization~Embedded systems</concept_desc>
%  <concept_significance>500</concept_significance>
% </concept>
% <concept>
%  <concept_id>10010520.10010575.10010755</concept_id>
%  <concept_desc>Computer systems organization~Redundancy</concept_desc>
%  <concept_significance>300</concept_significance>
% </concept>
% <concept>
%  <concept_id>10010520.10010553.10010554</concept_id>
%  <concept_desc>Computer systems organization~Robotics</concept_desc>
%  <concept_significance>100</concept_significance>
% </concept>
% <concept>
%  <concept_id>10003033.10003083.10003095</concept_id>
%  <concept_desc>Networks~Network reliability</concept_desc>
%  <concept_significance>100</concept_significance>
% </concept>
%</ccs2012>
%\end{CCSXML}

%\ccsdesc[500]{Computer systems organization~Embedded systems}
%\ccsdesc[300]{Computer systems organization~Redundancy}
%\ccsdesc{Computer systems organization~Robotics}
%\ccsdesc[100]{Networks~Network reliability}

%%
%% Keywords. The author(s) should pick words that accurately describe
%% the work being presented. Separate the keywords with commas.
\keywords{datasets, neural networks, gaze detection, text tagging}


\maketitle

%!TEX root = main.tex
\section{Introduction}\label{sec:introduction}

People search engines are increasingly common for job recruiting and even for finding companionship or friendship.
%
A top-$k$ ranking algorithm is typically used to find the most suitable way of ordering items (persons, in this case), considering that if the number of people matching a query is large, most users will not scan the entire list.
%
Conventionally, these lists are ranked in descending order of some measure of the relative fitness of items, according to the \emph{probability ranking principle}~\cite{robertson1977probability}.

The main concern motivating this paper is that a machine learning ranking model may produce ranked lists that can systematically reduce the visibility of an already disadvantaged group~\cite{peder2008,Dwork2012}.
%
Disadvantaged groups correspond to legally protected categories such as people with disabilities, racial or ethnic minorities, or an under-represented gender in a specific professional domain.
%
Furthermore it is assumed that this bias manifests differently across groups, rendering inter-group relevance scores incomparable with each other.


According to \citet{friedman1996bias} a computer system is \emph{biased} ``if it systematically and unfairly discriminate[s] against certain individuals or groups of individuals in favor of others.
%
A system discriminates unfairly if it denies an opportunity or a good or if it assigns an undesirable outcome to an individual or a group of individuals on grounds that are unreasonable or inappropriate.''
%
Yet ``unfair discrimination alone does not give rise to bias unless it occurs systematically'' and ``systematic discrimination does not establish bias unless it is joined with an unfair outcome.''
%
On a ranking, the desired good for an individual is to appear in the result and to be ranked amongst the top-$k$ positions.
%
The outcome can be deemed unfair if members of one or more protected groups are systematically ranked lower than those of a privileged group.
%
The ranking algorithm discriminates unfairly if this ranking decision is based fully or partially on the protected feature.
%
This discrimination is systematic when it is embodied in the algorithm's ranking model.
%
As shown in earlier research, a machine learning model trained on datasets incorporating \textit{preexisting bias} will embody this bias and therefore produce biased results, potentially increasing any disadvantage further, reinforcing existing bias~\cite{oneil2016weapons}.


Based on this observation, in this paper we study the problem of producing a ranking that we will consider fair given legally-protected attributes.
%
A formal definition \textcolor[rgb]{0.00,0.00,1.00}{of the problem studied} can be found in Section~\ref{sec:problem}: \textcolor[rgb]{0.00,0.00,1.00}{intuitively, our aim is} to produce a ranking in which the proportion of different minority groups in any prefix of the ranking does not fall below minimum proportions $p_G$. In this ranking, we also would like to preserve relevance/utility as much as possible.
%
%We denote however the motivating assumption of our method which is that utility measures \emph{are not comparable across groups} because said bias manifests differently for each group.

We propose a \textcolor[rgb]{0.00,0.07,1.00}{re-ranking} method to remove the systematic bias by means of a \emph{ranked group fairness criterion}, that we introduce in this paper.
%
We assume a ranking algorithm has given an undesirable outcome to one ore more groups of individuals, but the algorithm itself cannot determine if the grounds were appropriate or not.
%
Hence we expect the user of our method to know that the outcome is based on unreasonable or inappropriate grounds and provide $p_G$ as input which can originate in a legal mandate or in voluntary commitments.
%
For instance, the US Equal Employment Opportunity Commission sets a goal of 12\% of workers with disabilities in federal agencies in the US,\footnote{US EEOC: \url{https://www1.eeoc.gov/eeoc/newsroom/release/1-3-17.cfm}, Jan 2017.}
%
while in Spain, a minimum of 40\% of political candidates in voting districts exceeding a certain size must be women~\cite{verge2010gendering}.
%
In other cases, such quotas might be adopted voluntarily, for instance through a diversity charter.\footnote{European Commission: \url{http://ec.europa.eu/justice/discrimination/diversity/charters/}}
%
In general these measures do not mandate perfect parity, as distributions of qualifications across groups can be unbalanced for legitimate, explainable reasons~\cite{zliobaite2011handling,pedreschi2009integrating}. % pedreschi2009integrating has the truck driver license example }


The ranked group fairness criterion compares the number of protected elements in every prefix of the ranking with the expected number of protected elements if they were picked at random using a multinomially distributed statistical process (``dice rolls'' with each side $ j $ of the dice having success probability $p_j \in p_G$).
%
Given that we use a statistical test for this comparison, we include a significance parameter $\alpha$ corresponding to the probability of a Type I error, which means rejecting a fair ranking in this test.

\begin{example} Consider the three rankings in Table \ref{tbl:multinomial_intro_example} corresponding to top 10 ranking applicants in a credit approval process. The rankings are obtained based on the credit worthiness of each applicant taking into different account features such as account status, credit duration andcredit amount. 
	We also present to which protected group each individual belonged w.r.t. their demographics features such as age and race. As observed, the above rankings show that 
	individuals belonged to protected groups have less exposure in the top 10 results than favored group which can lead to systematically disadvantages individuals of one protected group (e.g., younger than 30 or not-white) to others (between 30 and 50, older than 50 or white) in accessing credit. Additionally, individuals belonged to multiple protected groups (younger than 30 and not white) have the least exposure in the top 10 results.
	
	More specifically, we show if the first ranking in Table \ref{tbl:multinomial_intro_example} can pass the test of \textit{ranked group fairness} proposed in ~\cite{zehlike2017fair} for one protected group (younger than 25 here). Suppose that the required proportion for this group is $p = 0.4$ and significant value is $\alpha = 0.1$, this translates to having at least one individual from the protected group in the first 5 positions: therefore the first ranking in table 1 would be accepted as fair for young age group but would be rejected as unfair for not-white group and also young and not-white group for the same value of $p$ and $a$.
	
	Considering the second ranking in Table \ref{tbl:multinomial_intro_example} one not-white protected group and suppose that the required proportion for this group is $p = X$ and significant value is $\alpha = Y$. This translates to having at least one individual from the protected group in the first 6 positions: therefore the second ranking in table 1 would be accepted as fair for not-white group. However, assume that 
	the required proportion for young group is $p = 0.4$, the second ranking would be also accepted as fair for younger than 25 group, however, it would be rejected as unfair for not-white and young group for either vaalue of p.
	
	From the above we can conclude that providing the fair ranking results for each protected group can not guarantee the fair ranking for multiple protected groups. This highlights the need to ranked group fairness notion taking into account multiple protected groups (i.e., multinomial). As an example, the third ranking in Table \ref{tbl:multinomial_intro_example} is an example of the top 10 ranking that achieves the protection for multiple protected groups: young, not-white and also young and not-white using our \textit{multinomial ranked group fairness} criteria and algorithm proposed in this paper. 
	
\end{example}
\medskip

\begin{table}[t]
	\caption{Example of top-10 results for different protected groups in a credit approval process.
		%observe that the top-10 results under-represent the least represented gender, in comparison with top-40 results.
		%
		\label{tbl:multinomial_intro_example}}
	
	\centering\small\begin{tabular}{lcccc}\toprule
		& Position					  & top 10 & top 10  & top 10 \\
		%          & \multicolumn{1}{c}{Position}                      & 10   & 10     & 40 \\
		& \texttt{1 2 3 4 5 6 7 8 9 10} & young & not-white & y\& \\
		&                               &  (y)  & (nw)  &  nw \\
		\midrule
		Young  & \texttt{o/w o/w o/w y/w o/w o/w o/w o/w o/w o/w} & 10\% & 0\% & 0\% \\
		not-white & \texttt{o/w o/w o/w y/w o/w o/nw o/w o/w o/w o/w} & 10\% & 10\% & 0\% \\
		young and not-white & \texttt{o/w o/w o/w o/w o/w y/nw o/w o/w o/w o/w} & 20\% & 10\% & 10\% \\
		\bottomrule
	\end{tabular}
	
\end{table}


\medskip
\begin{example} Consider the three rankings in Table \ref{tbl:xing_intro_example} corresponding to searches for an ``economist,'' ``market research analyst,'' and ``copywriter'' in XING\footnote{\url{https://www.xing.com/}}, an online platform for jobs that is used by recruiters and headhunters, mostly in German-speaking countries, to find suitable candidates in diverse fields (data collection is reported in detail in~\cite{zehlike2017fair}). While analyzing  the extent to which candidates of both sexes are represented as we go down these lists,  we can observe that such proportion keep changing and is not uniform (see, for instance, the top-10 vs. the top-40). As a consequence, recruiters examining these lists will see different proportions depending on the point at which they decide to stop.
%
This outcome systematically disadvantages individuals of one sex by preferring the other at the top-$k$ positions, a clear instance of bias in a computer system~\cite{friedman1996bias}. As we do not know the learning model behind the ranking, we assume that the result is at least partly based on the protected attribute \emph{sex}.

\note{Why do we need this assumption? Imposing group-fairness constraint makes sense in any case. This assumption sounds like a not-needed justification to me.}

Let $k = 10$. Our notion of \textit{ranked group fairness} imposes a fair representation with proportion $p$ and significance $\alpha$ at each top-$i$ position with $i \in [1,10]$ (formal definitions are given in section~\ref{sec:problem}).
Consider for instance $\alpha = 0.1$ and suppose that the required proportion is $p = 0.4$.  This translates (see Table \ref{tbl:ranked_group_fairness_table}) to having at least one individual from the protected minority class in the first 5 positions: therefore the ranking for  ``copywriter'' would be rejected as unfair. However, it also requires to have at least 2 individuals from the protected group in the first 9 positions: therefore also the ranking for ``economist'' is rejected as unfair, while the ranking for ``market research analyst'' is fair for  $p = 0.4$. However, if we would require $p = 0.5$ then this translates in having at least 3 individuals from the minority group in the top-10, and thus even the ranking for ``market research analyst'' would be considered unfair.
%
We note that for simplicity, in this example we have not adjusted the significance $\alpha$ to account for multiple statistical tests; this is not trivial, and is one of the key contributions of this paper.
\end{example}
\medskip

\begin{table}[t]
	\caption{Example of non-uniformity of the top-10 vs. the top-40 results for different queries in XING (Jan 2017).
		%``economist,'' ``market research analyst,'' and ``copywriter'' in XING (Jan 2017). We %do not claim anything about the merits of these proportions, but simply
		%observe that the top-10 results under-represent the least represented gender, in comparison with top-40 results.
		\label{tbl:xing_intro_example}}

	\centering\small\begin{tabular}{lccccc}\toprule
		& Position					  & top 10 & top 10  & top 40 & top 40 \\
		%          & \multicolumn{1}{c}{Position}                      & 10   & 10     & 40   & 40 \\
		& \texttt{1 2 3 4 5 6 7 8 9 10} & male & female & male & female \\
		\midrule
		Econ.  & \texttt{f m m m m m m m m m} & 90\% & 10\% & 73\% & 27\% \\
		Analyst& \texttt{f m f f f f f m f f} & 20\% & 80\% & 43\% & 57\% \\
		Copywr.& \texttt{m m m m m m f m m m} & 90\% & 10\% & 73\% & 27\% \\
		\bottomrule
	\end{tabular}


\end{table}

\todo{We would need another example with multiple protected attributes.}



We remark that our method is suitable for concerns about \emph{intersectionality}, which refers to the fact that \textcolor[rgb]{0.00,0.00,1.00}{the personal}, political, and social identities \textcolor[rgb]{0.00,0.00,1.00}{of an individual can be combined to form a unique profile that can be discriminated}.
%
Consider as an example the case DeGraffenreid v. General Motors in 1976: here Emma DeGraffenreid wanted to challenge General Motors for being discriminated in the promotion procedure for being a black woman.
%
However the court found that GM had neither discriminated against blacks, nor against women and therefore ruled that DeGraffenreid ``could not combine the claims'' of race and sex discrimination.
%
Yet many cases of intersectional discrimination demand for a method like ours that is capable of dealing with biases against groups that are oppressed based on more than one protected \textcolor[rgb]{0.00,0.00,1.00}{attribute}.

\textcolor[rgb]{0.00,0.00,1.00}{\subsection{Contributions and Roadmap}}
\textcolor[rgb]{0.00,0.00,1.00}{This work presents an extension of our previous paper~\cite{zehlike2017fair}, which was published at the CIKM 2017 conference. In this extended version} we define and analyze the {\sc Fair Top-$k$ Ranking problem} with multinomial groups, in which we want to determine a subset of $k$ candidates from a large pool of $n \gg k$ candidates, in a way that maintains high utility (selects the ``best'' candidates from each group), subject to group fairness criteria.
%
The running example we use in this paper is that of selecting automatically, from a large pool of potential candidates, a smaller group that will be interviewed for a position.

Our notion of utility assumes that we want to invite the most qualified candidates from each group, while their qualification is equal to a relevance score calculated by a ranking algorithm.
%
This score is assumed to be based on relevant metrics for evaluating candidates for a position, which depending on the specific skills required for the job could be their grades (e.g., Grade Point Average), their results in a standardized knowledge/skills test specific for a job, their typing speed in words per minute for typists, or their number of hours of flight in the case of pilots.
%
We note that this measurement will embody \emph{pre-existing bias} (e.g. if black pilots are given less opportunities to flight they accumulate less flight hours), as well as \emph{technical bias}, as learning algorithms are known to be susceptible to direct and indirect discrimination~\cite{tuto2016,HajianFerrer12}.
%
We furthermore note that different manifestations of such bias exists for each group and are usually stronger for intersectional groups, i.e. the pre-existing bias against black women is stronger than the one for women or blacks in general.

This utility principle is operationalized in two ways.
%
First, by a criterion we call \emph{selection utility}, which prefers rankings in which every candidate included in the top-$k$ is more qualified than every candidate not included, or in which the difference in their qualifications is small.
%
Second, by a criterion we call \emph{ordering utility}, which prefers rankings in which for every pair of candidates included in the top-$k$, either the more qualified candidate is ranked above, or the difference in their qualifications is small.
%
Note however, that in a setting with multiple protected groups optimal selection and ordering utility cannot be guaranteed because of said differences in the group skews.
%
Mathematically this means that the optimal solution for multinomial ranked group fairness (i.e. for more than one protected group) is a solution space rather than just a single point as it was in~\cite{zehlike2017fair}, while the optimal solution in terms of utility is still a single point within said solution space whose location depends on the candidate set at hand.
%
We want to stress that trying to find this point of optimal utility corresponds to a worldview in which one assumes that utility measures of candidates across different groups are actually comparable and that the per-group bias is known a-priori.
%
We believe that the group skew unawareness is a necessary condition for the justification of post-processing algorithms in general and we therefore explicitly do not search for the optimal solution in terms of utility.
%
We will go into more depth on this in Section~\ref{subsec:individual-fairness}.

Our definition of \emph{ranked group fairness} reflects the legal principle of group under-representation in obtaining a benefit \cite{ellis2012eu,lerner2003group}.
%
We use the standard notion of protected groups (e.g., ``people with disabilities''); where protection emanates from a legal mandate or a voluntary commitment.
%
%The group under-representation principle, and the related disparate impact doctrine~\cite{Barocas2014} addresses the fact that there might be differences in qualification across different groups by \emph{not} mandating an equal proportion of candidates from the protected group and non-protected group in the output. It simply states that the proportions cannot be too different.
%
We formulate a criterion by applying a statistical test on the proportion of protected candidates on every prefix of the ranking, which should be indistinguishable or above a given minimum.
%
%This procedure can be seen as a form of positive action to ensure that the proportion of protected candidates in every prefix of the top-$k$ ranking is statistically indistinguishable from a policy target.
%
We also show that the verification of the ranked group fairness criterion can be implemented efficiently by pre-computing a verification data structure that we call \emph{mTree}.
%
This tree contains all possibilities to create a ranking that satisfies ranked group fairness and we provide an algorithm to build and persist it.
%
We also provide an algorithm \algoCorrect for the mTree adjustment due to multiple dependent hypothesis testing, where we compute $\alphaadj$ such that the type-I-error is less or equal to $\alpha$ for the \emph{entire tree}.

Finally, we propose an efficient algorithm, named \algoFAIR, for producing a top-$k$ ranking that maintains high utility while satisfying ranked group fairness, as long as there are ``enough'' protected candidates from each group to achieve the desired minimum proportions.
%
We also present extensive experiments to evaluate the performance of our approach compared to the so-called ``color-blind'' ranking with respect to both the expected utility of a ranking and the fairness degree measured in terms of expected exposure.

\medskip

Summarizing, the main contributions of this paper are:
\begin{compactenum}
	\item the principled definition of \emph{ranked group fairness} for multiple protected groups, and the associated  {\sc Fair Top-$k$ Ranking problem};
	\item the \algoComputeMTree algorithm to provide the mTree, a data structure that \algoFAIR needs as input
	\item the \algoCorrect algorithm to find the adjusted mTree, which has a maximum probability for a type-1-error of $\alpha$
	\item the \algoFAIR\ algorithm for producing a top-$k$ ranking that maximizes utility while satisfying ranked group fairness.
\end{compactenum}

Our method can be used within an anti-discrimination framework such as \emph{positive actions}~\cite{sowell2005affirmative}.
%
We do not claim these are the only way of achieving fairness, but we provide \emph{an algorithm grounded in statistical tests that enables the implementation of a positive action policy in the context of ranking}.


\todo{Summarise in details the differences between the conference version and this extended one.}



The rest of this paper is organized as follows.

\todo{The roadmap below should be revised and extended a bit once we're totally sure about the structure of the paper.}

The next section presents a brief survey of related literature, while Section~\ref{sec:problem} introduces our ranked group fairness and utility criteria, our model adjustment approach, and a formal problem statement.
%
Section~\ref{sec:multinom-fair-algo} describes the \algoFAIR\ algorithm and the model adjustment algorithm.
%
Section~\ref{sec:experiments} presents experimental results.
%
Section~\ref{sec:conclusions} presents our conclusions and future work.

%!TEX root = main.tex
\section{Related Work}\label{sec:related-work}

%Discrimination analysis is a multi-disciplinary problem, involving sociological causes, legal reasoning, economic models, statistical techniques~\cite{Romeimulti}.
Anti-discrimination has only recently been considered from an algorithmic perspective~\cite{tuto2016}. Some proposals are oriented to discovering and measuring discrimination ({\em e.g.}, \cite{peder2008,Bonchi2015,angwin_2016_machine}); while others deal with mitigating or removing discrimination to ensure that the results of different algorithms do not lead to discriminatory decisions
%even if the training dataset reflects the historical bias against members of a protected group
({\em e.g.}, \cite{CaldersICDM,HajianFerrer12,hajian2014,Dwork2012,Zemel2013}).
%
All these methods are known as \emph{fairness-aware algorithms}.

\subsection{Group fairness and individual fairness}
Two basic frameworks have been adopted in recent studies on algorithmic discrimination: \begin{inparaenum}[(i)]
	\item \emph{individual fairness}, a requirement that individuals should be treated consistently~\cite{Dwork2012, zliobaite2015survey}; and
	\item \emph{group fairness}, also known as statistical parity, a requirement that the protected groups should be treated similarly to the advantaged group or the population as a whole \cite{peder2008,pederruggi2009}.
\end{inparaenum}

Different fairness-aware algorithms have been proposed to achieve group and/or individual fairness, mostly for predictive tasks. \citet{Calders2010} consider three approaches to deal with naive Bayes models by modifying the learning algorithm.
%, two of which consist in modifying the learning algorithm: training a separate model for each protected group; and adding a
%latent variable to model the class value in the absence of discrimination.
\citet{CaldersICDM} modify the entropy-based splitting criterion in decision tree induction to account for attributes denoting protected groups.
\citet{Kamishima2012}  apply a regularization ({\em i.e.}, a
change in the objective minimization function) to probabilistic discriminative models, such
as logistic regression. \citet{zafar2015} describe fairness constraints for several classification methods.

\citet{Feldman2015} study \emph{disparate impact} in data, which corresponds to an unintended form of group discrimination, in which a protected group is less likely to receive a benefit than a non-protected group~\cite{Barocas2014}.
%
A technically similar framework has been proposed by~\citet{zehlike2020matching} for a setting with multiple protected groups, which continuously interpolates between group fairness as statistical parity and individual fairness.
%
We use this method as one of our experimental baselines in \S\ref{sec:experiments-baselines}.
%
Their method uses optimal transport to calculate a fair score representation for each group, which is a the Wasserstein-barycenter of all group distributions.
%
It also enables a ``continuous interpolation'' that generates protected group scores between the original distribution in the protected groups and the distribution of the barycenter; this is controlled by a parameter but there are no guarantees on the proportions along the ranking.
%
Recently, other fairness-aware algorithms have been proposed for mostly supervised learning algorithms and different bias mitigation strategies \cite{hardt2016equality, jabbari2016fair, friedler2016possibility, celis2016fair, corbett2017algorithmic}.
%
\citet{hardt2016equality} study fairness in terms of \emph{equalized odds} and proposes methods to ensure equal precision across groups.
%
\citet{zafar2017fairness} introduces a new concept called \emph{disparate mistreatment} and proposes a method that seeks to reduce differences in error rates.

In contrast, our paper considers the more general setting by achieving fairness in ranking algorithms and it even can be reduced to other problems such as classification (see Section~\ref{concept:related-problems}). Our framework provides an evaluation criterion, ranked group fairness, and offers guarantees under this criterion, which can be directly controlled by a external parameter. Moreover, it takes into account both frameworks of group and individual fairness.

\subsection{Fair Ranking}
%
Algorithmic fairness for rankings is concerned with a sufficient representation of different groups and consistent treatment of individuals across all ranking positions~\cite{castillo2018fairness}.
%
This leads to the motivation of producing rankings that do not systematically place items from protected groups at the lower positions of the result list.

\citet{yang2016measuring} propose a statistical parity measure based on comparing the distributions of protected and non-protected candidates on different prefixes of the list and then averaging these differences in a discounted manner.
%
%The discount used is logarithmic, similarly to Normalized Discounted Cumulative Gain (NDCG, a popular measure used in Information Retrieval~\cite{jarvelin2002cumulated}).
%
%Finally, they show very preliminary results on incorporating their statistical parity measure into an optimization framework for improving fairness of ranked outputs while maintaining accuracy.
%
We use the synthetic ranking generation procedure of~\citet{yang2016measuring} to calibrate our method, and optimize directly the utility of a ranking that has statistical properties (ranked group fairness) resembling the ones of a ranking generated using that procedure; in other words, unlike them, we connect the creation of the ranking with the metric used for assessing fairness.
Moreover, we also take into account the individual fairness criteria.
%
Yang {\em et al.} \citet{yang2018nutritional} present a collection of metrics for fairness in ranked outputs which is inspired by nutrition labels.
%
Kulshrestha {\em et al.} \citet{kulshrestha_2017_quantifying} proposes a quantification framework that measures the bias of the results in a search engine.
This framework discerns to what extent this output bias is due to the input dataset that feeds into the ranking system, and how much is due to the bias introduced by the system itself.
%
Kuhlman {\em et al.} \citet{kuhlman2019fare} also propose a quantification and diagnostics framework using pairwise error metrics as means of evaluation for fairness in rankings.
%
Another quantification and bias mitigation framework is presented by~\citet{geyik2019fairness}, which can be tuned towards different definitions of fairness such as demographic parity and so-called equal opportunity.

Celis {\em et al.} \citet{celis2017ranking} propose algorithms for constrained ranking, in which the constraint is a $k \times \ell$ matrix with $k$ being the length of the ranking and $\ell$ the number of classes, indicating the maximum number of elements of each class (protected or non-protected in the binary case) that can appear at any given position in the ranking.

Singh and Joachims \citet{singh2018fairness} introduce the concept of exposure of a group that incorporates the empirical observation that the probability for items to be examined by a searcher decreases quickly with each lower position.
%
Parallel with~\citet{biega2018equity}, they propose an integer linear program that receives a vector of relevance scores and ranks items accordingly subject to constraints in disparate attention across groups. They also introduce a time component into their optimization problem, such that their notion of disparate attention is to be considered over time.

Lahoti {\em et al.} \citet{lahoti2019operationalizing} focus on \emph{individual fairness} for rankings and propose an advancement of its definition by~\citet{Dwork2012} in such way that it no longer relies on a human specification of a distance metric.

\citet{zehlike2018reducing} introduce the first bias mitigation algorithm within a learning to rank framework, by defining the learning objective in terms of accuracy as well as in terms of exposure fairness of the result ranking.
% RIC: I thought that was Joachims in WSDM 2017
%
The learning objective is concerned with disparate impact, whereas~\citet{beutel2019fairness} and~\citet{singh2019policy} introduce fairness objectives in learning to rank methods that comply with disparate treatment considerations.

\subsection{Diversity}

To avoid showing only items of the same class has been studied in the Information Retrieval community for many years. The motivation there is that the user query may have different intents and we want to cover several of them with the answers.
%
The most common approach, since \citet{carbonell1998use}, is to consider distances between elements, and maximize a combination of relevance (utility) with a penalty for adding to the ranking an element that is too similar to an element already appearing at a higher position.
%
Another often-used definition is that diversity should be understood as a way of incorporating uncertainty over user intents, in the sense that all queries have some degree of ambiguity~\cite{agrawal2009diversifying}.
%
Other works \citet{kunaver2017diversity,channamsetty2017recommender} deal with different kinds of bias such as presentation bias, where, only a few items are shown and most of the items are not shown, and also popularity bias and a negative bias towards new items.
%
An exception is \citet{sakai2011evaluating}, that provides a framework per-intent for evaluating diversity, in which an ``intent'' could be mapped to a protected/non-protected group in the fairness ranking setting. Their method, however, is concerned with evaluating a ranking, similar to the NDCG-based metrics described by~\citet{yang2016measuring} and that we described before, and not with a construction of such ranking, as we do in this paper.

In contrast with most of the research on diversity of ranking results or recommendation systems, our work operates on a discrete set of classes (not based on similarity to previous items).
%
Furthermore, \emph{we are not only concerned with the utility that search system users receive}, but also with the exposure of the items being ranked, which can represent individual, organizations, or places.
%
Another key difference is that diversification is usually symmetric yielding interchangeable groups, while \emph{fairness-aware algorithms are usually asymmetric}, as they focus on increasing the overall benefit received by a disadvantaged or protected group.

%!TEX root = main.tex

\section{The Fair Top-k Ranking Problem}\label{sec:problem}

In this section, we first present the needed notation (\S\ref{subsec:preliminaries}), then the ranked group fairness criterion (\S\ref{subsec:group-fairness}) and criteria for utility (\S\ref{subsec:individual-fairness}). Finally we provide a formal problem statement (\S\ref{subsec:problem-statement}).
%
This formalization extends the one previously presented for a binomial setting (one protected group)~\cite{zehlike2017fair} into a multinomial setting.

\subsection{Preliminaries and Notation}
\label{subsec:preliminaries}
\textcolor[rgb]{0.00,0.00,1.00}{We consider a set of candidates} $[n] = \{ 1, 2, \dots, n \}$  out of which $k$ candidates will be selected. \textcolor[rgb]{0.00,0.00,1.00}{We are also given} $q_c$ for \textcolor[rgb]{0.00,0.00,1.00}{each} $c \in [n]$, denoting the ``utility'' of candidate $c$: this can be interpreted as an overall summary of the fitness of candidate $c$ for the specific job, task, or search query, and it could be obtained by the combination of several different attributes.
If this utility is computed by a machine learning model, an effort must be done to prevent preexisting and technical biases with respect to a protected group to be embodied in the model (see, e.g., \cite{Sweeney2013}).
%
We define a set of candidate groups $G = \left\{g_0, g_1, \ldots\textcolor[rgb]{0.00,0.00,1.00}{ g_{|G|}}\right\}$ that form a partition of $[n]$. We will consider $g_0$ as non-protected (or privileged) and all remaining groups as protected (or disadvantaged).
%
To simplify the presentation of the algorithms, we will assume there are at least $k$ candidates in each group. % enough candidates of each group, i.e., at least $k$ of each kind.

Let ${\mathcal P}_{k,n}$ represent all the subsets of $[n]$ containing exactly $k$ elements.
%
Let ${\mathcal T}_{k,n}$ represent the union of all permutations of sets in ${\mathcal P}_{k,n}$, i.e., the solution space of the top-$k$ ranking problem.
%
For a permutation $\tau \in {\mathcal T}_{k,n}$ and an element $c \in [n]$, let
\[
r(c, \tau) = \begin{cases}
\mathrm{rank~of~} c \mathrm{~in~} \tau & \mathrm{if~} c \in \tau~, \\
%|\tau| + 1 & \mathrm{otherwise}.
k + 1 & \mathrm{otherwise}.
\end{cases}
\]

We further define $\tau_g$ to be the number of elements of group $ g $ that are present in $\tau$, i.e. $\tau_g = | \{ c \in \tau \wedge c \in g \} |$.
%
We set $ \tau_G = \langle\tau_g\rangle_{g \in G}$, i.e., the vector that contains these numbers for each group.
%

\textcolor[rgb]{0.00,0.00,1.00}{Let $cb \in {\mathcal T}_{n,n}$ be the total ranking of all candidates by
 decreasing utility: $\forall u,v \in [n], u=cb(i), v = cb(j), i < j \implies q_u \ge q_v$.}
%
For simplicity of exposition, we will assume all utilities $q_c$ are different, although our algorithms do not require this to be the case.
%
We call this the \emph{color-blind} ranking of elements in $[n]$, because it simply focuses on the utility and ignores whether elements are protected or non-protected.
%
Let $\textit{cb}|_k = \langle \textit{cb}(1), \textit{cb}(2), \ldots, \textit{cb}(k) \rangle$ be a prefix of size $k$ of this ranking. \label{concept:color-blind-ranking} 
%

\begin{table}[t]
\caption{Notation.}
\CaptionMargin
\label{tbl:notation}
\small
\begin{tabular}{R{.2\linewidth}L{.75\linewidth}}\toprule
\multicolumn{2}{c}{Candidates} \\
\midrule
$[n]$ & Set of candidates \\
$q_c$ & Qualifications of candidate $c$ \\
$g_c \in G$ & 0 if candidate $c$ is in the non-protected group, $ >0 $ otherwise\\
$|G|$ & The number of protected groups \\
\midrule
\multicolumn{2}{c}{Rankings} \\
\midrule
${\mathcal T}_{k,n}$ & All permutations of $k$ elements of $[n]$ \\
$\tau$ & One such permutation \\
$r(c,\tau)$ & The position of candidate $c$ in $\tau$, or $|\tau|+1$ if $c \notin \tau$ \\
$ \tau_G~=~\left(\tau_1, \tau_2, \ldots, \tau_{|G|}\right)$ & Vector of number of elements from group $ g $ in $\tau$ \\
$\textit{cb}$ & The ``color-blind'' ranking of $[n]$ by decreasing $q_c$ \\
\midrule
\multicolumn{2}{c}{Group fairness criteria} \\
\midrule
$p_G~=~\left(p_1, p_2, \ldots, p_{|G|}\right)$ & Vector of minimum proportions for candidates of each protected group $ g > 0 $ \\
$\alpha$ & Significance value for ranked group fairness test \\
$\alphaadj$ & Adjusted significance for each fair representation test \\
\midrule
\multicolumn{2}{c}{Individual fairness criteria} \\
\midrule
$	\texttt{utility}(c,\tau)$ & Qualification difference between candidate $c$ and the least qualified candidate ranked above $c$ while $q_c > q_d$ \\
\midrule
\multicolumn{2}{c}{Model Adjustment} \\
\midrule
$ \failprob $ & Probability that our test fails on a fair ranking \\
$ m_{\alpha, p}(k)$ & Minimum number of protected candidates in top $k$ positions; a vector of integers in case of $|G| > 1$ \\
\bottomrule
\end{tabular}
\tablemargin
\end{table}

\spara{Fair top-$k$ ranking criteria.}\label{concept:criteria}
We would like to obtain $\tau \in {\mathcal T}_{k,n}$ with the following objectives, which we describe formally next: %\S\ref{subsec:group-fairness} and \S\ref{subsec:individual-fairness}:

\begin{enumerate}[{Criterion} 1.]
	\item Ranked group fairness: $\tau$ should fairly represent each protected group; \label{cond:ranking}

	\item Expected selection utility: $\tau$ should contain the most qualified candidates; and \label{cond:selection}

	\item Expected ordering utility: $\tau$ should be ordered by decreasing qualifications.\label{cond:ordering}
\end{enumerate}

We will provide a formal problem statement in \S\ref{subsec:problem-statement}, but first, we need to provide a formal definition of each of the criteria, which we do in the next sections.
\textcolor[rgb]{0.00,0.00,1.00}{The notation used in this paper is summarized on Table~\ref{tbl:notation}.}


\subsection{Group Fairness for Rankings}
\label{subsec:group-fairness}

We operationalize Criterion~\ref{cond:ranking} of Section~\ref{concept:criteria} by means of a \emph{ranked group fairness criterion}, which takes as input
\begin{inparaenum}[(i)]
	\item $ \tau_G $, the vector containing the number of candidates from each protected group in ranking $ \tau $, and
	\item $ p_G $, a vector containing minimum target proportions for each protected group.
\end{inparaenum}
Intuitively, this criterion declares the ranking as unfair if candidates in a protected group is far below the required number according to the target proportions.
%
Additionally, this criterion looks at the ordering in which those candidates appear.
%
Specifically, the ranked group fairness criterion compares the number of protected elements from each group \emph{in every prefix} of the ranking, with the expected number of protected elements if they were picked at random using a stochastic process with a multinomial distribution, such as the roll of a dice.

\begin{definition}[Multinomial Cumulative Distribution Function]
	\label{def:multinomialCDF}
	% We do not need to cite this -- ChaTo
  %	Adopting the notation from~\cite{multinomcdf}
	Let $ n \in \mathbb{N}$ be a number of trials where each trial results in one of the events $ E_1, E_2, \ldots, E_{|G|} $ and on each trial $ E_j $ occurs with probability $ p_j $.
	%
	Let then $X$ %=\left\{X_1, X_2, \ldots, X_k\right\} $
	be a set of random variables that is multinomially distributed $ X \sim \operatorname{Mult}(n, p)$ with parameters $ n $ and $ p = \langle p_1, p_2, \ldots, p_{|G|} \rangle$, and let $ X_j $ be the number of trials in which event $ E_j $ occurs.
	%
	We then define $ F\left(X; n, p\right) = P\left(E_1 \leq X_1, E_2 \leq X_2, \ldots, E_{|G|} \leq X_{|G|}\right)$ the multinomial cumulative distribution function which computes the probability that each event $ E_j $ occurs at most $ X_j $ times in $ n $ trials given probabilities $ p $.
\end{definition}
With the multinomial CDF the ranked group fairness criterion is formulated as a statistical significance test, and we include a significance parameter ($\alpha \textcolor[rgb]{0.00,0.00,1.00}{\in [0,1]}$) corresponding to the probability of rejecting a fair ranking (i.e., a Type I error).

\begin{definition}[Fair representation condition]
	\label{def:fair-representation-condition}
	% Using n and p because they're standard notation for binomials
	Let $F(X;n,p)$ be the multinomial cumulative distribution function as defined above.
	%
	A set $\tau \subseteq \mathcal{T}_{k,n}$, having $\tau_G$ protected candidates from each group fairly represents all protected groups with minimal proportions $p_G = (p_1, p_2, \ldots, p_{|G|})$ and significance $\alpha$,
	%
	if $F(\tau_G;k,p_G) > \alpha$.
\end{definition}

This is equivalent to using a statistical test where the null hypothesis $H_0$ is that the protected elements of each group are represented with a sufficient proportion $p_t$ ($\forall g \in G, p_t \ge p_g$), and the alternative hypothesis $H_a$ is that the proportion of protected elements is insufficient ($\exists g \in G: p_t < p_g$). In this test, the p-value is $F(\tau_G; k, p_G)$ and we reject the null hypothesis, and thus declare the ranking as unfair, if the p-value is less than or equal to the threshold $\alpha$.
%
Note that according to this definition, in the case of a set of size one, either the element is in the protected group, and then we satisfy fair representation, or the element is not in the protected group, and then we satisfy fair representation if $1 - F > \alpha$.

The ranked group fairness criterion enforces the fair representation condition over all prefixes of the ranking:

\begin{definition}[Ranked group fairness condition]
	\label{def:ranked-group-fairness-condition}
	A ranking $\tau \in {\mathcal T}_{k,n}$ satisfies the ranked group fairness condition with parameters $p_G$ and $\alpha$, if for every prefix $\tau|_i = \langle \tau(1), \tau(2), \dots, \tau(i) \rangle$ with $1 \le i \le k$, the set $\tau|_i$ satisfies the fair representation condition with group target proportions $p$ and significance $\alphaadj = \adj(\alpha, k, p_G)$.
	%
	Function $\adj(\alpha, k, p_G)$ is a corrected significance to account for multiple hypotheses testing (described in Section~\ref{sec:model-adjustment}).
\end{definition}

We note \textcolor[rgb]{0.00,0.00,1.00}{there exists} a solution space of rankings that satisfy this condition for a given 3-tuple $(k, p_G, \alpha)$, instead of just a single ranking as it is the case when only one protected group is present.
We further note that a larger $\alpha$ means a larger probability of declaring a fair ranking as unfair.
%
In our experiments (Section~\ref{sec:experiments}), we use a relatively conservative setting of $\alpha=0.1$.
%
The ranked group fairness condition can be used to create a \emph{ranked group fairness measure}. For a ranking $\tau$ and probabilities $p_G$, the ranked group fairness measure is the maximum $\alpha \in [0,1]$ for which $\tau$ satisfies the ranked group fairness condition.
%
Larger values indicate a stricter adherence to the required number of protected elements at each position.

\subsection{Utility}
\label{subsec:individual-fairness}
Our notion of utility reflects the desire to select candidates that are potentially better qualified, and to rank them as high as possible.
%
In contrast with previous works~\cite{yang2016measuring,celis2017ranking}, we do not assume to know the utility contribution of a given candidate at a particular position, but instead we base our utility calculation on losses due to non-monotonicity (i.e., due to candidates not being ordered by decreasing scores anymore to satisfy the ranked group fairness constraint).
%
This can also be understood as a measure of individual unfairness, as it calculates the largest score difference between a high-scoring candidate ranked below a low-scoring candidate.

The qualifications may have been even proven to be biased against protected groups, as is the case with the COMPAS scores~\cite{angwin_2016_machine} that we use in the experiments of Section~\ref{sec:experiments}, but our approach can bound the effect of that bias, because the utility maximization is subject to the multinomial ranked group fairness constraint.
%
However, we denote again that due to different manifestations of bias we believe that measures of qualification are not comparable across groups of candidates.
%
This means that for the entire fair solution space we can guarantee maximized utility only within groups but not across and we want to stress again that maximizing utility over the entire set of candidates comes with a change in one's believe system in which candidate utilities are indeed comparable.
%
In such a case we would question the usage of a post-processing algorithm in the first place.
%
However we want to operationalize the selection and ordering utility criterion from Section~\ref{subsec:preliminaries} and provide means for the user of our algorithm to estimate the loss of ranking relevance w.r.t. the colorblind ranking and hence define the following utility measures.

\note[Meike]{The whole paragraph above is unclear. It should be rephrased. 
	Relate to old paper and say that optimality proof is no longer possible. Give black and white females example. } 

\spara{Ranked utility.}
The ranked individual utility associated to a candidate $c$ in a ranking $\tau$, compares it against the least qualified candidate ranked above it.

\begin{definition}[Ranked utility of an element]
	\label{def:rankedIndividualFairness}
	The ranked utility of an element $c \in [n]$ in ranking $\tau$, is:
	\[
	\texttt{utility}(c,\tau) = \begin{cases}
	\overline{q} - q_c &\textcolor[rgb]{0.00,0.00,1.00}{ \textrm{if~} \overline{q} < q_c \textrm{where~} \overline{q}\triangleq \min_{d: r(d,\tau) < r(c,\tau)} q_d } \\
	0 & \textrm{otherwise}\\
	\end{cases}
	\]
\end{definition}
%
\noindent By this definition, the maximum ranked individual utility that can be attained by an element is zero. %\note{It's weird to have something called utility which has its maximum in zero. It would have been more appropriate to call it something like \emph{individual unfairness}, but I guess it's too late to change terminology (same applies to several other notions).}
%
%Next, we apply the definition of ranked individual utility to two separate cases: when an element $i$ is included in the ranking, and when it is not included.

\spara{Selection utility.}
%
We operationalize Criterion~\ref{cond:selection} by means of a \emph{selection utility} objective, which we will use to prefer rankings in which the more qualified candidates are included, and the less qualified, excluded.
%
\begin{definition}[Selection utility]
	\label{def:selectionFairness}
	The selection utility of a ranking
	$\tau \in {\mathcal T}_{k,n}$ is \[\min_{c \in [n], c \notin \tau} \texttt{utility}(c,\tau)\].
\end{definition}
%
\noindent Naturally, a ``color-blind'' top-k ranking $\textit{cb}|_k$ maximizes selection utility, i.e., has selection utility zero.

\spara{Ordering utility and in-group monotonicity.}
%
We operationalize Criterion~\ref{cond:ordering} of Section \ref{concept:criteria} by means of an \emph{ordering utility} objective and an \emph{in-group monotonicity constraint}, which we will use to prefer top-$k$ lists in which the more qualified candidates are ranked above the less qualified ones.

\begin{definition}[Ordering utility]
	\label{def:orderingFairness}
	The ordering utility of a ranking $\tau \in {\mathcal T}_{k,n}$ is \[\min_{c \in \tau} \texttt{utility}(c,\tau)\].
\end{definition}

\noindent The ordering utility of a ranking is only concerned with the candidate attaining the worst (minimum) ranked individual utility. Instead, the in-group monotonicity constraints refer to all elements, and specifies that within groups candidates must be sorted by decreasing qualifications.

\begin{definition}[In-group monotonicity]
	\label{def:inGroupMonotonicity}
	A ranking $\tau \in {\mathcal T}_{k,n}$ satisfies the in-group monotonicity condition if $\forall c,d$ s.t. $g_c = g_d$, $r(c,\tau) < r(d,\tau) \Rightarrow q_c \ge q_d$.
\end{definition}

\noindent Again, the ``color-blind'' top-k ranking $\textit{cb}|_k$ maximizes ordering utility, i.e., has ordering utility zero; it also satisfies the in-group monotonicity constraint.

\spara{Connection to the individual fairness notion.}\label{concept:our-utility-individual-fairness}
%
Our notion of utility is centered on individuals, for instance by taking the minima instead of averaging.
%
While other choices are possible, this has the advantage that we can trace loss of utility to specific individuals. These are the people who are ranked below a less qualified candidate, or excluded from the ranking, due to the ranked group fairness constraint.
%
This is connected to the notion of individual fairness, which requires people to be treated consistently~\cite{Dwork2012}. Under this interpretation, a consistent treatment should require that two people with the same qualifications be treated equally, and any deviation from this is in our framework a utility loss. This allows trade-offs to be made explicit.

\subsection{Formal Problem Statement}
\label{subsec:problem-statement}
The criteria we have described allow for different problem statements, depending on whether we use ranked group fairness as a constraint and maximize ranked utility, or vice versa.
%In this paper, we study in depth the following problem statement (an algorithm is presented in Section~\ref{sec:algorithms}).

\newtheorem*{problem*}{Problem}
\begin{problem*}[Fair top-k ranking]
	Given a set of candidates $[n]$, \textcolor[rgb]{0.00,0.00,1.00}{a partition of $[n]$ in groups $G = \left\{g_0, g_1, \ldots, g_{|G|}\right\}$, the vector $p_G$ of minimum proportions per group,  and parameters $k \in \mathbb{N}^+$ and $\alpha \in [0,1]$}, produce a ranking $\tau \in {\mathcal T}_{k,n}$ that:
	\begin{compactenum}[(i)]
		\item \label{problem:constraint-monotonicity} satisfies the in-group monotonicity constraint;
		\item \label{problem:constraint-rank} satisfies ranked group fairness with parameters $p_G$ and $\alpha$;
		\item \label{problem:optimal-sel} achieves high selection utility subject to (\ref{problem:constraint-monotonicity}) and (\ref{problem:constraint-rank}); and
		\item \label{problem:maximum-ord} achieves high ordering utility subject to (\ref{problem:constraint-monotonicity}), (\ref{problem:constraint-rank}), and (\ref{problem:optimal-sel}).
	\end{compactenum}
\end{problem*}

\spara{Related problems.}\label{concept:related-problems}
%
Alternative problem definitions are possible with the general criteria described in Section~\ref{concept:criteria}.
%
For instance, instead of maintaining high selection and ordering utility, we may seek to always find the one ranking $\tau^*$ from all possible rankings that satisfy ranked group fairness that maximizes selection and ordering utility.
%
We acknowledge that this seems to be a tempting ``optimization'' of \algoFAIR but we believe that such a strategy is not fully compliant with multinomial ranked group fairness anymore.
%
As ranked group fairness is defined through a statistical process that corresponds to the roll of a dice, all possible ``fair'' solutions must have a chance $>0$ to be picked as the final result ranking.
%
As soon as $\tau^*$ becomes the only solution the algorithm finds, we will render true ranked group fairness obsolete.
%
Consider as a real world example again the case of different biases across groups in an intersectional setting:
%
Assuming we have a dataset with white, upper-class and black, lower-class people in which the utility criterion is lightly correlation with gender and strongly correlating with income class.
%
The protected group would be white women, black men and black women.
%
If we were to always choose the ranking that satisfies ranked group fairness and maximizes utility, we would \emph{always} reorder the result in favor of white women, while black men and particularly black women would fall behind.
%
Instead \algoFAIR should truly implement ranked group fairness in which all solutions (also those that favor blacks) have a certain chance to be accepted.

%!TEX root = main.tex

\section{Ranked Group Fairness Verification}
\label{sec:fairness-verification}
%!TEX root = main.tex

\begin{figure}[t!]
	\centering
	\resizebox{1\columnwidth}{!}{%
	\begin{forest}
		for tree={
			child anchor=west,
			parent anchor=east,
			grow'=east,
			draw,
			anchor=west,
		}
		[{1, [0, 0]}
		[{2, [0, 0]}
			[{3, [1, 0]}
				[{4, [2, 0]$^*$}
					[{5, [3, 0]}
						[{6, [3, 1]}
							[{7, [3, 1]}, name=doubled5
								[{8, [4, 1]}
									[{9, [5, 1]}]
									[{9, [4, 2]}]
								]
								[{8, [3, 2]}, name=parentDoubled8]
							]
						]
					]
					[{5, [2, 1]}, name=parentDoubled2]
				]
				[{4, [1, 1]}, name=doubled1
					[{5, [1, 1]}
						[{6, [2, 1]}, name=doubled2
							[{7, [2, 2]}, name=doubled6, before drawing tree={y-=1em}
								[{8, [2, 2]}, before drawing tree={y-=1em}
									[{9, [3, 2]}, name=doubled8, before drawing tree={y-=1em}]
									[{9, [2, 3]}, name=doubled9, before drawing tree={y-=1em}]
								]
							]
						]
						[{6, [1, 2]}, name=doubled3]
					]
				]
			]
			[{3, [0, 1]}, name=parentDoubled1
				[{4, [0, 2]}
					[{5, [1, 2]}, name=parentDoubled3]
					[{5, [0, 3]}
						[{6, [1, 3]}
							[{7, [1, 3]}, name=doubled7
								[{8, [2, 3]}, name=parentDoubled9]
								[{8, [1, 4]}
									[{9, [2, 4]}]
									[{9, [1, 5]}]
								]
							]
						]
					]
				]
			]
		]]
		\draw (parentDoubled1.east)--(doubled1.west);
		\draw (parentDoubled2.east)--(doubled2.west);
		\draw (parentDoubled3.east)--(doubled3.west);
		\draw (doubled2.east)--(doubled5.west);
		\draw (doubled3.east)--(doubled6.west);
		\draw (doubled3.east)--(doubled7.west);
		\draw (parentDoubled8.east)--(doubled8.west);
		\draw (parentDoubled9.east)--(doubled9.west);
	\end{forest}
	}
	\CaptionMargin
	\caption{Example of an mTree with two protected groups with minimum proportions $ p_G=\langle 1/3, 1/3 \rangle $ and $ \alpha=0.1 $. The notation $k, [x,y]$ indicates that at ranking position $k$, we need at least $x$ elements of group $1$ and $y$ elements of group $2$ to satisfy ranked group fairness; group 0 which is the non-protected group is always unconstrained.
	%
	%Levels go from left to right starting from 1, as indicated by the first number in the nodes. 
	For instance, the node marked ``4, [0,2]$^*$'' indicates that when 4 elements are present (level 4), one of the acceptable configurations is to have 2 or more elements from group 1 and 0 or more elements from group 2.
	%
	We see that in case of multiple protected groups, there are various ways of satisfying Definition~\ref{def:ranked-group-fairness-condition}.
	%
	Thus, each path in the tree corresponds to one valid strategy to place protected candidates in the ranking.
	\label{fig:mtree-symmetric-unadjusted}}
\end{figure}
%
%\begin{figure}[t!]
%	\centering
%	\begin{forest}
%		for tree={
%			child anchor=west,
%			parent anchor=east,
%			grow'=east,
%			draw,
%			anchor=west,
%		}
%		[{[0, 0]}
%		[{[0, 0]}
%			[{[1, 0]}
%				[{[2, 0]}, before drawing tree={y-=1em}
%					[{[2, 1]}, before drawing tree={y-=1em}
%						[{[2, 1]}, before drawing tree={y-=1em}
%							[{[2, 1]}, name=doubled5, before drawing tree={y-=2em}
%								[{[2, 2]},  before drawing tree={y-=2em}
%									[{[2, 2]}, name=doubled6, before drawing tree={y-=2em}
%										[{[3, 2]},  before drawing tree={y-=2em}]
%									]
%								]
%							]
%						]]]
%				[{[1, 1]}
%				[{[1, 1]}, name=doubled1
%					[{[1, 1]}, name=parentDoubled5
%						[{[1, 2]}, name=doubled2, before drawing tree={y-=1em}
%							[{[1, 2]}, name=parentDoubled6, before drawing tree={y-=1em}
%								[{[1, 3]}, name=doubled3, before drawing tree={y-=2em}
%									[{[2, 3]},  before drawing tree={y-=2em}]
%									[{[1, 4]}, name=doubled4, before drawing tree={y-=2em}]
%								]
%							]
%						]
%					]]]
%			]
%			[{[0, 1]}
%				[{[0, 1]}, name=parentDoubled1
%					[{[0, 2]}
%						[{[1, 2]}, name=parentDoubled2]
%						[{[0, 3]}
%							[{[0, 3]}
%								[{[1, 3]}, name=parentDoubled3]
%								[{[0, 4]}
%									[{[1, 4]}, name=parentDoubled4]
%									[{[0, 5]}
%										[{[1, 5]}]
%									]
%								]
%							]
%						]
%					]
%				]]
%		]]
%		\draw (parentDoubled1.east)--(doubled1.west);
%		\draw (parentDoubled2.east)--(doubled2.west);				
%		\draw (parentDoubled3.east)--(doubled3.west);
%		\draw (parentDoubled4.east)--(doubled4.west);				
%		\draw (parentDoubled5.east)--(doubled5.west);
%		\draw (parentDoubled6.east)--(doubled6.west);
%	\end{forest}
%	\CaptionMargin
%	\caption{Example of an mTree with two protected groups with minimum proportions $ p_G=\langle 0.2, 0.4 \rangle $ and $ \alpha=0.1 $. We see that the tree, in contrast to the mTree in Figure~\ref{fig:mtree-symmetric-unadjusted}, is not symmetric because the minimum proportions $ p_G $ differ.
%	\label{fig:mtree-asymmetric-unadjusted}}
%\end{figure}

To verify ranked group fairness efficiently in time $O(k)$, a pre-computed data structure can be used, which is obtained by the \emph{inverse multinomial CDF} with parameters $k, p_G$ and $ \alpha $.
%
The inverse CDF specifies the value of the random variable, such that the probability of this variable being less than or equal to that value equals a given probability (in our case $p_G$).
%
As the multinomial CDF is not injective and has hence no inverse, there is no quantile function that tells us exactly how many candidates are needed at each $ k $.
%
Instead, there are various manifestations of $ \tau_G $ that satisfy the fair representation condition $F(\tau_G;k,p_G) > \alpha$, which is why the verification data structure has the shape of a tree for each $ p_G, k $ and $ \alpha $.
%
Figure~\ref{fig:mtree-symmetric-unadjusted} shows an example of such tree with $p_G = [1/3, 1/3]$. 
%
Each tree level corresponds to the $k$-th position in the ranking and are to be read from left to right, i.e. the root level corresponds to the first ranking position and so forth.
%
As an example, consider tree level 4 in Fig.~\ref{fig:mtree-symmetric-unadjusted}.
%
At this level, we have three nodes ``[2,0],'' ``[1,1],'' and ``[0,2],'' each of them containing a set of minima for elements of the protected groups (the nodes do not include any minima for the non-protected group).
%
This means it is acceptable to have among the first four elements in the ranking either at least 2 elements from protected group 1, or at least 1 element from each protected group, or at least 2 elements from protected group 2.
%
Note however, that nodes have parental relationships and that each \emph{path} corresponds to a fair distribution of protected candidates in the ranking.
%
Thus, if at level 4 we rank two candidates from protected group 1, thus satisfying the node with the asterisk in Fig.~\ref{fig:mtree-symmetric-unadjusted}, at level 5 have to satisfy either node ``[2,1],'' or node ``[3,0].''
%
The other nodes ``[1,1],'' ``[1,2],'' and ``[0,3]'' are not a child of node ``[2,0]'' and therefore cannot be considered to satisfy the ranked group fairness condition anymore.

The tree is symmetric when the minimum proportions are equal. 
%
Figure~\ref{fig:mtree-asymmetric-unadjusted} show the tree is asymmetric when the minimum proportions are different, in that example $0.2$ and $0.4$.

%%%%%%%%%%%%%%%%%%%%%%%%%%%%%%%%%%%%%%%%%%%%%%%%%%%%%%%%%%%%%%%
% ALGORITHM COMPUTE MTREE
%%%%%%%%%%%%%%%%%%%%%%%%%%%%%%%%%%%%%%%%%%%%%%%%%%%%%%%%%%%%%%%
\begin{algorithm}[h]
	\caption{Algorithm \algoComputeMTree computes the data structure to efficiently verify or construct a ranking that satisfies multinomial ranked group fairness.}
	\label{alg:computeMTree}
	\small
	\AlgInput{$k$, the size of the ranking to produce; $p_G$, the expected proportions of protected elements for each group; $\alphaadj$, the significance for each individual test.}
	\AlgOutput{$ \mtree $: A tree data structure that contains the minimum number of protected candidates for each group.}
	$\mtree[0] \leftarrow \text{zeros(G)}$ \AlgComment{initialize auxiliary root node with $ G $ entries} 
	\For{$i \leftarrow 1$ \KwTo $k$}{
		\For{\tt parent in $ \mtree[i - 1] $}{
			\AlgComment{find all child nodes that satisfy ranked group fairness}
			$\tt children \leftarrow \algoImcdf(\alphaadj; i, p_G, \tt parent)$ \\
			\tt parent.children $ \leftarrow $ \tt children\\
		}
	}
	\Return{$ \mtree $ }
\end{algorithm}

\begin{algorithm}[h]
	\caption{Algorithm \algoImcdf computes the inverse of the multinomial cumulative distribution function $ F^{-1}(\alphaadj; i, p_G) $. It finds all possible child nodes of a given parent that satisfy the ranked group fairness condition. }
	\label{alg:imcdf} 
	\small
	\AlgInput{ \texttt{parent}, the node of which we calculate all minimum target children; \\
		$i$, the current position in the ranking; \\
		$p_G$, the vector of expected proportions of protected elements of each group;\\
		$\alphaadj$, the significance for each individual test.}
	\AlgOutput{ \texttt{children}: A list of nodes with minimum targets that satisfy ranked group fairness}
	\tt children $\leftarrow \lbrace \rbrace$ \\
	\tt child $ \leftarrow  $ \tt copy(parent) \\
	\tt mcdf $ \leftarrow  F(\texttt{child}; i, p_G) $ \\
	\If{\tt mcdf $ > \alphaadj $}{
	\AlgComment{if the multinomial cdf is greater than $\alphaadj$ we do not need to increase the number of required protected candidates}
		\tt children.add(child)
	} \Else {
		\For{$j \leftarrow 1 $ to $ |G|$}{
		\AlgComment{test whether the multinomial cdf is greater than $\alphaadj$, if one more candidate of group $j$ was required at postion $i$}
			\tt temp[j] $\leftarrow$ temp[j] + 1\\
			\tt mcdfTemp $\leftarrow F(\texttt{temp}; j, p_G)$ \\
			\If{$\texttt{mcdf} > \alpha_c$}{
			\AlgComment{if yes, append the new requirement to the mTree}
				children.add(temp)			
			}
		}
	}
	\Return{children}
\end{algorithm}

As stated at the beginning of this section mTrees are pre-computed data structures that allow efficient verification of the ranked group fairness condition.
%
To construct them we use algorithms~\ref{alg:computeMTree} and~\ref{alg:imcdf}.
%
The first algorithm \algoComputeMTree takes a triple $(k,p_G,\alphaadj)$ as input and returns the mTree for these parameters.
%
First it creates an auxiliary root node that contains $|G|$ zero entries, which serves as the root parent.
%
Then for each parent node \texttt{parent} at each position position $i \leq k$ it calls the inverse multinomial CDF function.
%
The second algorithm \algoImcdf takes as input a 4-tuple $(\alphaadj, i, p_G, \texttt{parent})$ and returns all nodes satisfying the following two conditions:
\begin{inparaenum}[(i.)]
	\item they have to be children of node \texttt{parent}, and
	\item they satisfy the fair representation condition (Def.~\ref{def:fair-representation-condition}, page~\pageref{def:fair-representation-condition}).
\end{inparaenum}

\section{Model Adjustment}
\label{sec:model-adjustment}

\note[ChaTo]{This is a long subsection ... I wonder if it can be split into two or three subsections, e.g., 4.3 Model Adjustment 4.4 Optimizations, or something like that.

In general this section \ref{sec:alpha-adjustment-multi} deals with two different topics: one is how to adjust the $\alpha$, the other one (necessary for adjusting $\alpha$ but also for creating a fair ranking) is about optimizing the mTree generation and storage. The latter could include material currently in \ref{subsec:algorithm-optimization}.

A division between these two should be clearer.}

Our ranked group fairness definition requires an adjusted significance $\alphaadj = \adj(\alpha, k, p)$, because it tests multiple hypotheses ($k$ of them).
%
If we use $\alphaadj = \alpha$, we will have too many false positives, i.e., we reject rankings that are fair, at a rate larger than $\alpha$.
%
To adjust the significance of multinomial \algoFAIR, we extend the generative model for fair rankings by~\citet{yang2016measuring} which we used in~\cite{zehlike2017fair}, to use a multinomial distribution instead of a binomial one.
%
Specifically, we consider that the following ranking should be accepted by the ranked group fairness test: \begin{inparaenum}[(i)]
	\item start with an empty list, and
	\item incrementally add the next best protected candidate from group $g$ with probability $p_g$, or the next best available non-protected candidate with probability $p_0 = 1-\sum_{j=1}^{|G|} p_j$.
\end{inparaenum}
%
``Next best'' means the one with the highest utility.

\begin{figure}[h]
	\centering
	{\includegraphics[width=.48\textwidth]{pics/failProbPlotMultinom.png}}
	\caption{Experiments on data generated by a simulation, showing the need for multiple tests correction.
		%
		The data has two protected groups, rankings are created by a multinomial process (``rolling a 3-sided dice'') with $p_G = (0.33, 0.33)$.
		%
		These rankings should have been rejected as unfair at a rate $\alpha = 0.1$.
		%
		However, we see that the rejection probability increases with $k$.
		%
		Note the scale of $k$ is logarithmic.}
	\label{fig:why-adjustment-is-needed-multinomial}
\end{figure}
To illustrate why this correction is needed, observe Figure~\ref{fig:why-adjustment-is-needed-multinomial}, which assumes two protected groups with $p_1=p_2=\frac{1}{3}$.
%
This figure is generated by simulation, generating rankings with the process described above.
%
It shows the probability of those rankings being rejected by our ranked group fairness test with $\alphaadj=0.1$.
%
Generally, we can see that the probability of a Type-I error (declaring this fair ranking as unfair) is higher than $\alpha = 0.1$.
%
Therefore, depending on $k$ we would need to change the value of $\alpha$, if we want to achieve a rejection rate of $0.1$.
%
If the $k$ tests were independent, we could use $\alphaadj = 1 - (1 - \alpha)^{1/k}$ (i.e., {\v S}id{\'a}k's correction), but given the positive dependence, the false negative rate is smaller than the bound given by {\v S}id{\'a}k's correction.

\subsection{General Procedure}
\label{subsec:general-process}
%
With the presence of more than one protected group, the analytical extension of the model adjustment to a multinomial setting is too complex to be written into a closed formula.
%
In contrast to the model adjustment for one protected group which we used in~\cite{zehlike2017fair} (see also Section~\ref{sec:adjustment-binomial}), we found no analytical way to calculate all permutations which pass or fail the test.
%
Therefore we develop an experimental procedure to adjust $ \alpha $:
%
\begin{enumerate}
	\item Get input $ p_G, k, \alpha $.
	\item Build mTree with input $ \alpha $.
	\item Create $M$ rankings by rolling a biased $ |G| $-sided dice with each side's probability to show corresponding to a minimum proportion in vector $ p_G $.
	\item Test all those rankings against the mTree and count how many tests fail.
	%
	Remember that we want to observe a maximum failure probability of $ \failprob=\alpha $, because all rankings created by this multinomial stochastic process are considered to be inherently fair.
	\item If $ \failprob \neq \alpha $, we choose a new $ \alphaadj $ using a binary search heuristic.
	\item Now we build a new mTree using $ \alphaadj $ and repeat the procedure until $ \failprob \approx \alpha $.
\end{enumerate}
%
Once we found $\alphaadj$ we can recompute the mTrees from Figures~\ref{fig:mtree-symmetric-unadjusted} and~\ref{fig:mtree-asymmetric-unadjusted} to obtain an overall significance level of $\alpha = 0.1$.
%
Figures~\ref{fig:mtree-symmetric-adjusted} and~\ref{fig:mtree-asymmetric-adjusted} show the adjusted mTrees with the same parameters $p_G$.
\begin{figure}[h]
	\centering
	\begin{forest}
		for tree={
			child anchor=west,
			parent anchor=east,
			grow'=east,
			draw,
			anchor=west,
		}
		[{[0, 0]}
		[{[0, 0]}
		[{[0, 0]}
		[{[0, 0]} 
		[{[0, 0]}
			[{[1, 0]}
			[{[1, 0]}
				[{[2, 0]}
				[{[2, 0]}
					[{[3, 0]}]
					[{[2, 1]}]
				]]
				[{[1, 1]}, name=doubled1
				[{[1, 1]}
				[{[1, 1]}
				]]]
			]]
			[{[0, 1]}
			[{[0, 1]}, name=parentDoubled1
				[{[0, 2]}
				[{[0, 2]}
					[{[1, 2]}]
					[{[0, 3]}]
				]]
			]]
		]]]]]]
		\draw (parentDoubled1.east)--(doubled1.west);
	\end{forest}
	\CaptionMargin
	\caption{Example of an mTree with two protected groups with minimum proportions $ p_G=[1/3, 1/3] $ and $ \alphaadj=0.1 $. Compared to figure~\ref{fig:mtree-symmetric-unadjusted} this tree is less strict such that its \emph{total} probability $ \alphaadj $ of rejecting a fair ranking (i.e. a type-1-error) is 0.1.
	\label{fig:mtree-symmetric-adjusted}}
	\tablemargin
\end{figure}

\begin{figure}[h]
	\centering
	\begin{forest}
		for tree={
			child anchor=west,
			parent anchor=east,
			grow'=east,
			draw,
			anchor=west,
		}
		[{[0, 0]}
			[{[0, 0]}
				[{[0, 0]}
					[{[1, 0]}
						[{[1, 0]}
							[{[2, 0]}
								[{[2, 1]}
									[{[2, 1]}, name=doubled2, before drawing tree={y-=1em}
										[{[2, 1]}, before drawing tree={y-=1em}
											[{[3, 1]}, before drawing tree={y-=1em}]
											[{[2, 2]}, before drawing tree={y-=1em}]
										]
									]
								]
							]
							[{[1, 1]}, name=doubled1
								[{[1, 1]}, name=parentDoubled2
									[{[1, 2]}, name=doubled3, before drawing tree={y-=1em}
										[{[1, 2]}, before drawing tree={y-=1em}
											[{[1, 2]}, before drawing tree={y-=1em}]
										]
									]
								]
							]
						]]							
					[{[0, 1]}
						[{[0, 1]}, name=parentDoubled1
							[{[0, 2]}
								[{[0, 2]}, name=parentDoubled3
									[{[0, 3]}, before drawing tree={y-=1em}
										[{[0, 3]}, before drawing tree={y-=1em}
											[{[1, 3]}, before drawing tree={y-=1em}]
											[{[0, 4]}, before drawing tree={y-=1em}]
										]
									]
								]
							]
						]]
					]]]
		\draw (parentDoubled1.east)--(doubled1.west);
		\draw (parentDoubled2.east)--(doubled2.west);
		\draw (parentDoubled3.east)--(doubled3.west);
	\end{forest}
	\CaptionMargin
	\caption{Example of an mTree with two protected groups with minimum proportions $ p_G=[0.2, 0.4] $ and a corrected $ \alpha_c=0.1 $. 
	%
	The tree is less strict than the mTree in Figure~\ref{fig:mtree-asymmetric-unadjusted}.
	%
	The adjusted tree yields an overall false-negative rate of $ \alpha_c=0.1 $ when testing rankings for ranked group fairness.
	\label{fig:mtree-asymmetric-adjusted}}
	\tablemargin
\end{figure}

\subsection{Optimizations for MTree Calculation}
\label{subsec:mtree-optimization}
In this subsection we explain how we optimize the adjustment procedure to reduce complexity, as it requires to compute a new mTree at each iteration which is expensive on its own. 
%
Furthermore, depending on the level of accuracy needed, a large number of iterations $M$ might be required and the binary search heuristic may need many steps to find $\alpha$, if large intervals have to be searched.
%
We use three strategies to drastically reduce computational costs of this simulation-based adjustment:
%
\begin{inparaenum}[(i)]
%
\item we reduce the space requirements of the mTree structure;
%
\item we construct the tree level by level and reject if the ranking does not satisfy any node in a level;
%
\item we exploit the monotonicity of $\alphaadj$ with respect to $\alpha$ and use a regression procedure to speed up the binary search.
\end{inparaenum}

\subsubsection{Reducing mTree Space Requirements.} 
We improve the mTree data structure by excluding the calculation and storage of redundant information.
%
First we store each node only once at each level and duplicate nodes are combined into a single node with multiple parents.
%
Furthermore we reduce the size of the mTree by actually leaving out the parent-child relationship and merely storing the nodes itself together with their respective depth levels.
%
Figures~\ref{fig:mtree-symmetric-adjusted} and~\ref{fig:mtree-asymmetric-adjusted} show the mTree structure with all parent-child relations as edges.
%
We leave out these edges, thus reducing space, because we can prove that, if a single node exists on each level, which accepts a given ranking as fair, then a valid path to that node exists in the tree.

Before we can prove this property, we need to introduce the following definition, which formalizes what a successful mTree test at level $i$ looks like.
%
\begin{definition}[Successful mTree Testing]
\label{def:valid-mtree-test}
Let $\tau$ be a ranking of size $k$ and $\tau_{G,i}=(\tau_{1},\ldots,\tau_{|G|})$ the numbers of ranked protected elements from group $1,\ldots,|G|$ up to position $i$.
%
Let furthermore $MT$ be a $mTree$, with $MT_{G,i}=[m_1(i),\ldots,m_{|G|}(i)]$ the number of protected candidates of group $1,\ldots,|G|$ required up to position $i$.
%
We write $\tau_{G,i} \geq m_{G,i}$ if $\tau_g \geq m_g$ for all $g=1,\ldots,|G|$.
%
We call a test on level $i$ of $MT$ successful, iff $\tau_{G,i} \geq m_{G,i}$.
\end{definition}
%
Again, in order to remove the parental relationships in the mTree, thus reducing storage space, we have to prove that, if we test a ranking on each level of the mTree successfully, the entire ranking will be fair according to the ranked group fairness definition.
%
We prove this by showing that, if a ranking passes the test for any two nodes $n_1$ and $n_2$ at two consecutive levels $h$ and $h+1$, and $n_1$ is \emph{not} a parent of $n_2$, then all actual children of $n_1$ will have a weaker requirement than $n_2$ and will hence also test successfully.
%
Furthermore we show that all nodes at level $h+1$ for which the ranking fails the test are part of a path that already rejected it as unfair at level $h$.
%
Consider an example from Figure~\ref{fig:mtree-symmetric-adjusted} : Let us assume a ranking passes the test at level $9$ with exactly the required protected items $[1,1]$.
%
Now lets assume that at level $10$, the given ranking would pass the test for node $[2,1]$, which is not a successor of $[1,1]$.
%
In fact we see that the actual successor of $[1,1]$ is a node with the same configuration $[1,1]$.
%
However, if our ranking passes the test for the stricter node $[2,1]$, it also passes for $[1,1]$ and thus we do not need to know the true parent of $[2,1]$.
%
Note that the ranking would fail at node $[3,0]$, but with $[1,1]$ at level 9 it would have failed already at $[3,0]$'s predecessor $[2,0]$.
%
\begin{theorem}
\label{theorem:lazy-mTree-test}
Let $MT$ be a mTree and $\tau$ a ranking of size $k$.
%
There exists at least one successful test for $\tau$ at each level of $MT$,
iff there exists a valid path from the root of $MT$ to a leaf of $MT$.
\end{theorem}
%
\begin{proof}
	\label{proof:lazy-mTree-test}
	It is clear that at least one successful test per level is necessary for the path to exist.
	%
	Let us proof that one successful test is a sufficient condition for the path to exist.
	%
	Let $MT$ be a mTree and $\tau$ be a ranking that passes the test at level $h$ of $MT$.
	%
	Let $m_{G,h}=[m_{1}(h), \ldots, m_{|G|}(h)]$ be the node on level $h$ that successfully tested $\tau$.
	%
	Let further be $\tau_g(h)$ the number of protected candidates of group $g$ ranked at up to position $h$.
	%
	Without loss of generality let $\sum_{g=1}^{|G|} |m_{g}(h) - \tau_{g}(h)| = 0$, meaning that the ranking includes the exact amount of required protected candidates at level $h$ and not more.
	%
	Let $m_{G,(h+1)}$ be a node which tests $\tau$ successfully on level $h+1$ with $\sum_{g=1}^{|G|} |m_{g}(h+1) - \tau_{g}(h+1)| = 0$.
	%
	For all entries $m_{g}(h+1)$ of $m_{G,(h+1)}$ it is that $m_{g}(h) \leq m_{g}(h+1)$ by construction of the mTree, in which requirements for protected candidates can only increase or stay the same, but cannot decrease.

	\noindent We can now distinguish between the following two cases:
	\\
	\textbf{Case 1:} $m_{G,(h+1)}$ is a child of $m_{G,h}$.
	%
	Then they form a path. %UNNECESSARY:% Note that $\sum_{g=1}^{|G|} (m_{g}(h+1) - m_{g}(h) \leq 1)$.
	\\
	\textbf{Case 2:} $m_{G,(h+1)}$ is not a child of $m_{G,h}$.
	%
	Let now ${m'}_{G,(h+1)}$ be a child of $m_{G,(h)}$.
	%
	Because of $\sum_{g=1}^{|G|} |m_{g}(h) - \tau_{g}(h)| = 0$, and a successful test at level $h+1$, the following inequations hold: $\sum_{g=1}^{|G|} |m_{g}(h) - m_g(h+1)| \leq 1$ and $\sum_{g=1}^{|G|} |m_{g}(h) - m'_g(h+1)| \leq 1$.
	%
	If $\sum_{g=1}^{|G|} |m_{g}(h) - m_g(h+1)| = \sum_{g=1}^{|G|} |m_{g}(h) - m'_g(h+1)| = 0$ or $\sum_{g=1}^{|G|} |m_{g}(h) - m_g(h+1)| = \sum_{g=1}^{|G|} |m_{g}(h) - m'_g(h+1)| = 1$, it follows that $m_{G,(h+1)} = {m'}_{G,(h+1)}$ and we would have a contradiction with the fact that ${m'}_{G,(h+1)}$ is not a child of $m_{G,h}$, because then the nodes would be equal or different by one unit in one position.

	Hence, we need that either
	(2.1) $\sum_{g=1}^{|G|} |m_{g}(h) - m_g(h+1)| = 1$ and $\sum_{g=1}^{|G|} |m_{g}(h) - m'_g(h+1)| = 0$,
	or
	(2.2) $\sum_{g=1}^{|G|} |m_{g}(h) - m_g(h+1)| = 0$ and $\sum_{g=1}^{|G|} |m_{g}(h) - m'_g(h+1)| = 1$.
	%
	Case (2.1) means because of $\sum_{g=1}^{|G|} |m_{g}(h) - m_g(h+1)| = 1 < \sum_{g=1}^{|G|} |m_{g}(h) - m'_g(h+1)| = 0$ that the mTree accepts a ranking with one more protected candidate than required by the actual child of $m_{G,h}$, which contradicts our hypothesis that the ranking included the exact amount of required protected candidates. It follows that if we test $\tau_g (h+1)$ successfully with $m_{G,h+1}$ it would also satisfy ${m'}_{G,h+1}$.
Case (2.2) is impossible according to algorithm \ref{alg:imcdf}.
In detail, if $\sum_{g=1}^{|G|} |m_{g}(h) - m_g(h+1)| = 0$ it means that $ m_g(h+1) = m_{g}(h)$ and therefore, $F(m_{g}(h);h+1,p_G) > \alpha$ (line $4$ of algorithm \ref{alg:imcdf}). But if for the child of $m_{G,h}$, namely ${m'}_{G,h+1}$ it holds that $\sum_{g=1}^{|G|} |m_{g}(h) - m'_g(h+1)| = 1$, it means that
$F(m_{g}(h);h+1,p_G) \leq \alpha$ so that we would have added a new node as a child of $m_{g}(h)$ according to lines  $8-14$ of algorithm \ref{alg:imcdf}. Since both conditions cannot be true at the same time, Case (2.2) can not occur.
\\
In summary, Case (2.1) will only occur if we ranked more protected candidates than needed, and that the resulting test would be more strict than following a path through the mTree.
%
We showed that Case (2.2) is impossible. There is only Case 1 left if we have ranked exactly the number of protected candidates needed at each level of the mTree.
%
It follows that for any ranking that is tested successfully on each level of the mTree, it either was tested by nodes of a path through the mTree or was tested by a series of nodes which is more strict than such path.
\end{proof}
%
\noindent Because of Theorem~\ref{theorem:lazy-mTree-test} we do not need to keep the tree structure (i.e. parental relationship between nodes) and may store only a set of nodes for each level, removing duplicate entries.
%
In case of symmetric minimum proportions $p_G$, we can further save memory and computation time, by reducing mirrored entries such as $[0,2]$ and $[2,0]$.
%
We simply flag a node for which a mirror exists on the same depth level and do not calculate the mirrored branch.

\subsubsection{Adjusting for Small $k$ First}
We use the fact that given a value of $\alpha$, the mTree calculation is not dependent on $ k $, i.e., a mTree for $k=20$ and a mTree for $k=10$ for the same $\alpha$ are equal in the first ten positions.
%
This means that we can start the adjustment from the root node and expand the tree gradually to find the correct $ \alphaadj $, because if $\failprob$ is too high for given $k$, it will also be too high for any $k' > k$. %in particular and we do not have to consider the larger tree, as long as we do not have a good $\alphaadj$ for $k=10$.
%
%In order to show that we can indeed do that, assume that the for loop of algorithm \ref{alg:computeMTree} stops at $k=10$ instead of $k=20$.
%
%Furthermore, we can understand the mTree as a set of rules that a ranking has to satisfy.
%
%If there are no rules for how many protected candidates are required after position $10$, this is equivalent to not requiring more protected candidates after position $10$.
%
%Thus, the probability that a fair ranking is rejected by a shorter mTree is less or equal to a deeper mTree.
%
%We utilize this property to lower computational costs: we calculate the mTree for $k=10$, then create 10000 rankings of length 10 and calculate $ \alphaadj $ under this setting.
%
%Then we set $ k=k+ $\texttt{stepsize}, and repeat the procedure until we reach the desired ranking length.
%
We can see in Figure~\ref{fig:why-adjustment-is-needed-multinomial} that $ \failprob $ grows very fast for small $k$, which makes an early adjustment of $\alpha$ most efficient to save computation time.

%
\begin{figure}[t!]
	\centering
	%
	\subfloat[Training data $R$ for a regression model to predict a good candidate for $\alphaadj$. Each pair $(k_j, \alpha_{c_j})$ is computed for small $k$ using the procedure described in Subsection~\ref{subsec:general-process}. \label{fig:regression-training-data}]
	{\includegraphics[width=.48\textwidth]{pics/alpha_030303_01.png}}\hfill
	%
	\subfloat[Computation time comparison between binary search only and binary search combined with regression for the multinomial significance adjustment. \inote{"regression" in the figure perhaps could be "regression and binary search"}  \label{fig:regression-time-saved}]
	{\includegraphics[width=.48\textwidth]{pics/computationTimeRegressionVSBinaryMultinomial.png}}\hfill
	\caption{}
	\label{fig:regression_adjustment_benefits}
	\vspace{-3mm}
\end{figure}
\subsubsection{Finding a Good Starting Candidate for the Binary Search}
We use a second-degree polynomial regression model to get our first estimate for a good $\alphaadj$ candidate and apply the binary search heuristic from that candidate, rather than starting with a random value, that might be very far away from the correct $\alphaadj$.
%
We need a few additional parameters as input: \texttt{kTarget} -- the length of the target ranking, \texttt{kStart} -- the size of the first mTree, \texttt{maxPreAdjustK} -- the maximum size of the mTree before we use regression to predict a good candidate $\alpha_{c_r}$ for the final $\alphaadj$, and \texttt{num\_iterations} -- the number of training instances to be computed.
%
To create a training dataset $R$, we compute \texttt{num\_iterations} small mTrees (i.e. with different $k \leq $ \texttt{maxPreAdjustK}) and adjust the respective $\alpha$ values as described in Subsection~\ref{subsec:general-process}.
%
For each iteration $j$ the pair $(k_j, \alpha_{c_j})$ is stored as a training instance in $R$.
%
Figure~\ref{fig:regression-training-data} shows a training set for $p_G=[1/3, 1/3], \texttt{maxPreAdjustK}=100$.
%
Then a regression model is trained to predict $\alpha_{c_r}$ for \texttt{kTarget}.
%
This $\alpha_{c_r}$ is now used to start the binary search for the correct and final $\alphaadj$.
%
Figure~\ref{fig:regression-time-saved} shows the runtime difference for the model adjustment routine with and without the use of regression.

\subsection{Final Adjustment Algorithm}
Algorithm~\ref{alg:regression_search} shows the overall adjustment algorithm in pseudo-code, which performs the following steps:
%
\begin{enumerate}
	\item Define the necessary parameters: \texttt{kTarget}, \texttt{kStart}, \texttt{maxPreAdjustK}, and \texttt{num\_iterations}
	\item Adjust $\alpha$ for a mTree of size \texttt{kStart} to get $\alpha_{c_1}$ using binary search.
	\item Add pair $\left(\texttt{kStart}, \alpha_{c_1}\right)$ to a regression training set $R$.
	\item \label{stepBegin} Increase $\texttt{kStart}$ by $\texttt{stepsize}=\frac{\texttt{maxPreAdjustK}}{\texttt{num\_iterations}}$
	\item Compute a mTree with parameters $\texttt{kTarget}, p_G, \alpha_{c_1}$ and adjust $\alpha_{c_1}$.
	\item \label{stepEnd} Name result $\alpha_{c_2}$ and add pair $\left(\texttt{kStart}, \alpha_{c_2}\right)$ to $R$.
	\item Repeat steps (\ref{stepBegin}) -- (\ref{stepEnd}) until $\texttt{kStart} == \texttt{maxPreAdjustK}$.
	\item Train a regression model with training data $R$ to predict $\alpha_{c_r}$ for $\texttt{kTarget}$.
	\item Use binary search (Algorithm~\ref{alg:mult_binary}) to find $\alphaadj$ for parameters $k,p_G, \alpha_{c_r}$.
\end{enumerate}
%
\begin{algorithm}[b!]
	\caption{Algorithm \algoReg estimates the corrected significance level $\alphaadj$ such that the mTree $m(\alphaadj , k, p_G)$ has the probability of rejecting a fair ranking $\alpha$}
	\label{alg:regression_search} % But whenever possible refer to this algo. by name not number
	\small
	\AlgInput{\texttt{kStart} -- depth of the mTree to start with; $k$ -- the length of the ranking; $p_G$ -- the desired proportions of the protected groups; $\alpha$ -- the desired significance level;  \texttt{maxPreAdjustK} the maximum depth of the mTrees that are used as training data; \texttt{num\_iterations} -- the number of steps between \texttt{kStart} and \texttt{maxPreAdjustK}}
	\AlgOutput{$\alphaadj$ -- the adjusted significance}
	$R \leftarrow \lbrace \rbrace; \alpha_{\textit{new}} \leftarrow \alpha$	\\
	\AlgComment{divide the interval [$\text{kStart}, \text{maxPreAdjustK}$] into num\_iterations parts}
	$\texttt{stepsize} \leftarrow \max(\frac{\texttt{maxPreAdjustK}}{\texttt{num\_iterations}}, 1)$ \\
	\For{$i\leftarrow 0$ to $\texttt{num\_iterations}$}{
		\AlgComment{adjust $\alpha_{\text{new}}$ for the current kStart}
    	$\alpha_{\text{new}} \leftarrow \textsc{MultinomialBinarySearchAdjustment}(\alpha_{\textit{new}}, \texttt{kStart}, p_G)$ \\
    	\AlgComment{add the pair (kStart, $\alpha_{\textit{new}}$) to the training data $R$}
    	$R.\textit{put}(\texttt{kStart},\alpha_{\textit{new}})$ \\
    	\If{$\texttt{kStart} + \texttt{stepsize} \leq \texttt{maxPreAdjustK}$}{
    		$\texttt{kStart} \leftarrow \texttt{kStart} + \texttt{stepsize}$ \\
    	}\Else{
			break
    	}
    }
    $\texttt{coeffs} \leftarrow R.\textit{train}()$ \AlgComment{returns the vector of predicted coefficients for the curve over $R$}
    $\alpha_{c_r} \leftarrow \texttt{coeffs[0]} + \texttt{coeffs[1]} * k + \texttt{coeffs[2]} * k^2$ \\
    $\alphaadj \leftarrow \textsc{MultinomialBinarySearchAdjustment}(\alpha_{c_r},k,p_G)$ \\
    \Return{$\alphaadj$}
\end{algorithm}

\FloatBarrier

\section{Algorithm}\label{sec:algorithms}
We present the multinomial \algoFAIR algorithm (\S\ref{subsec:algorithm-description}) and prove it is correct (\S\ref{subsec:algorithm-correctness}).

\subsection{Algorithm Description}\label{subsec:algorithm-description}
\note{Done}
Multinomial \algoFAIR, presented in Algorithm~\ref{alg:fair}, solves the {\sc Fair Top-$k$ Ranking} problem for multinomial protected groups and intersectional group settings.
%
As input, multinomial \algoFAIR takes 
the expected size $k$ of the ranking to be returned,
the qualifications $q_c$, 
indicator variables $g_c$ indicating if candidate $c$ is protected,
the vector of minimum target proportions $p_G$, and
the adjusted significance level $\alphaadj$.

First, the algorithm uses $q_c$ to create priority queues with up to $k$ candidates each: $P_0$ for the non-protected candidates and $P_g$ for the protected candidates of group $g$.
%
Next (line \ref{alg:fair:mtree}), the algorithm derives a ranked group fairness tree (mTree) similar to Figure~\ref{fig:mtree-asymmetric-adjusted}, i.e., for each position it computes the minimum number of protected candidates per group, given $p_G$, $k$ and $\alphaadj$.
%
Then, multinomial \algoFAIR greedily constructs a ranking subject to candidate qualifications, and minimum protected elements required.
%
Note that the right choice of a tree node on level $i$ depends on the path chosen along the tree and therefore on its concrete parent at level $i-1$. 
%
In case \texttt{mtree} branches into different possibilities to satisfy ranked group fairness (as an example see Fig.~\ref{fig:mtree-asymmetric-adjusted} for $k=4$), the algorithm chooses the branch that has a higher value for $F$, meaning it chooses the branch that has a higher probability.
%
If two branches are equally likely, which happens for $p_1 = p_2 = \ldots = p_{|G|}$, then one of them is chosen at random.
%
Given this the algorithm has to find the correct child $m_{G,i}$ for a given parent from the previous level (Line~\ref{alg:fair:childNode}).
%
If the node demands a protected candidate from group $g$ at the current position $i$, the algorithm appends the best candidate from $P_g$ to the ranking (Lines \ref{alg:fair:pstart}-\ref{alg:fair:pend}); otherwise, it appends the best candidate from $P_0 \cup P_1 \cup \ldots \cup P_{|G|}$ (Lines \ref{alg:fair:anystart}-\ref{alg:fair:anyend}).
%

\begin{algorithm}[h]
	%\caption{Algorithm \algoFAIR, finding a ranking that maximizes utility subject to in-group monotonicity and ranked group fairness constraints.}
	\caption{Algorithm \algoFAIR finds a ranking that maximizes utility subject to in-group monotonicity and ranked group fairness constraints. Checks for special cases (e.g., insufficient candidates of a class) are not included for clarity.}
	\label{alg:fair}  % But whenever possible refer to this algo. by name not number
	\small
	\AlgInput{$k \in [n]$, the size of the list to return; $\forall~c \in [n]$: $q_c$, the qualifications for candidate $c$, and $g_c$ an indicator that is $>0$ iff candidate $c$ is protected; $p_G$ with $\forall p \in p_G \in ]0,1[$, the vector of minimum proportions for each group of protected elements; $\alphaadj \in ]0,1[$, the adjusted significance for each fair representation test.}
	\AlgOutput{$\tau$ satisfying the group fairness condition with parameters $p, \sigma$, and maximizing utility.}
	%\AlgComment{compute min. protected candidates per position}
	$P_0, P_1, \ldots P_{|G|} \leftarrow$ empty priority queues with bounded capacity $k$\\
	\For{$c \leftarrow 1$ \KwTo $n$}{
		insert $c$ with value $q_c$ in priority queue $P_{g_c}$ \\
	}

	$\texttt{mtree}(i) \leftarrow \texttt{\algoComputeMTree}(k, p_G, \alphaadj)$  \label{alg:fair:mtree}\\
		
	%\AlgComment{create fair ranking}
	$(t_0, t_1, \ldots, t_{|G|}) \leftarrow (0, \ldots, 0)$ \\
	$i \leftarrow 0 $ \\
	\While{$i < k$}{
		\texttt{noCandidateAdded = True} \\
		\AlgComment{get next node in tree path}
		$m_{G, i} = [m_1(i), \ldots, m_{|G|}(i)] \leftarrow \texttt{findNextNode(mtree, i)}$ \label{alg:fair:childNode}\\
		\AlgComment{find which group needs a new candidate}
		\For{\texttt{g = 1; g} $\leq$ \texttt{|G|; g++}}{

			\If{$t_g < m_g(i)$}{\label{alg:fair:pstart}
				\AlgComment{add a protected candidate}
				$t_g \leftarrow t_g + 1$ \\ 
				$\tau[i] \leftarrow \operatorname{pop}(P_g)$ \\  
				\texttt{noCandidateAdded = False}
			}\label{alg:fair:pend}
		}
		\If{\texttt{noCandidateAdded}}{ \label{alg:fair:anystart}
			\AlgComment{no protected candidate needed: add the best available}
			$P_g \leftarrow$ \texttt{findBestCandidateQueue()} \\
			$\tau[i] \leftarrow \operatorname{pop}(P_g)$\\
			$t_g \leftarrow t_g + 1$ 
		}\label{alg:fair:anyend}
		
	}
	\Return{$\tau$}
\end{algorithm}
\vspace{-3mm}

\subsection{Algorithm Complexity}
\todo{Not done yet}

\begin{table}[]
\caption{Space and time complexity for all algorithms.
		\label{tbl:space_time}}
\begin{tabular}{|c|c|c|}
\hline
\textbf{Algorithm} & \textbf{Time Complexity} & \textbf{Space Complexity} \\ \hline
inverseBinomialCDF & $\mathcal{O}(n\log{}n)$ & $\mathcal{O}(n\log{}n)$ \\ \hline
\algoMtable & $\mathcal{O}(n\log{}n)$ & $\mathcal{O}(n\log{}n)$ \\ \hline
\algoRecursive & $\mathcal{O}(n\log{}n)$ & $\mathcal{O}(n\log{}n)$ \\ \hline
\algoBinomBinary & $\mathcal{O}(n\log{}n)$ & $\mathcal{O}(n\log{}n)$ \\ \hline
multinomialCDF & $\mathcal{O}(n\log{}n)$ & $\mathcal{O}(n\log{}n)$ \\ \hline
\algoImcdf & $\mathcal{O}(n\log{}n)$ & $\mathcal{O}(n\log{}n)$ \\ \hline
\algoComputeMTree & $\mathcal{O}(n\log{}n)$ & $\mathcal{O}(n\log{}n)$ \\ \hline
\algoMultBinary & $\mathcal{O}(n\log{}n)$ & $\mathcal{O}(n\log{}n)$ \\ \hline
\algoReg & $\mathcal{O}(n\log{}n)$ & $\mathcal{O}(n\log{}n)$ \\ \hline
\algoFAIR & $\mathcal{O}(n\log{}n)$ & $\mathcal{O}(n\log{}n)$ \\ \hline
\end{tabular}
\end{table}

\algoFAIR has running time $O(n + k \log k)$; which includes building the $O(k)$ size priority queues from $n$ items and processing them to obtain the final ranking, where we assume $k < O(n/\log n)$. 
\meike{Was soll denn $k < O(n/\log n)$ heißen? Warum ist die size der priority queues O(k) und nicht k?}
%
If we already have ranked lists for all groups of elements, \algoFAIR can avoid the first step and obtain the top-$k$ in $O(k \log k)$ time.
%
Our method is applicable as long as there is at least one protected group and there are enough candidates in each protected group; if there are $k$ from each group, the algorithm is guaranteed to succeed, otherwise the ``head'' of the ranking will satisfy the ranked group fairness constraint, but the ``tail'' of the ranking may not.

\subsection{Algorithm Optimizations}
\note{Done}
Because the computation of an adjusted mTree is expensive (Table~\ref{tbl:space_time}), our implementation persists already computed mTrees and their components to never do the same computation twice. 
%
Depending on the structure of $p_G$, different levels of optimization are applicable.

We note however that an mTree has to be computed only once for a particular combination of $k, p_G, \alpha$ and that \algoFAIR has a complexity of $O(k log k)$.
%
We provide the already pre-computed mTrees and MCDF caches for our experiments and the intermediate steps not only for reproducibility but for use in practice too.

\subsubsection{MCDF Cache}\label{subsubsec:mcdf-cache}
Table~\ref{tbl:space_time} shows that the highest computational cost arises from computing the multinomial cumulative distribution function $F$. 
%
In the worst case Algorithm~\ref{alg:imcdf} computes it $|G|+1$ times for each group $g$ in $m_g(i)$ and each position $i\leq k$.
%
However, the same calculation may be done many times:
%
As an example consider the (fictive) mTree nodes $[2,1]$ and $[1,2]$ at position $k=3$. 
%
To compute the successors of node $[2,1]$ we call Algorithm~\ref{alg:imcdf} with arguments $(4,[2,1])$, $(4,[3,1])$ and $(4,[2,2])$. 
%
We store the results of these calculation in a map that we call MCDF cache with the algorithm arguments ($k$ and the minimum protected candidates of each group) as key and the corresponding mcdf as value.
%
Next we compute the successors of node $[1,2]$ and call Algorithm~\ref{alg:imcdf} with arguments $(4,[1,2])$, $(4,[2,2])$ and $(4,[1,3])$. 
%
We see that we would compute the mcdf for $(4,[2,2])$ twice, but instead we can now read it from the MCDF cache.

Furthermore, if our example has symmetric minimum proportions $p_1 = p_1$, the mcdf of $(4,[2,1])$ is equal to $(4,[1,2])$ and $\textit{mcdf}(4,[1,3]) = \textit{mcdf}(4,[3,1])$. 
%
Generally, if $p_1 = p_2 = \cdots = p_{|G|}$, we can make use of the mTree's symmetry: we calculate the mcdf only for node $m(i)$ and store it in the cache \emph{as well as its mirror} (see Section~\ref{subsubsec:discarding-symmetric-nodes}), because their mcdf values are the same.

Note that the mcdf computation is only depends on $p_G$ and not on $alpha$. 
%
We can therefore persists the MCDF cache on disk for a particular vector $p_G$ and load it for any mTree calculation with the same $p_G$ in the future.
%
This also saves additional computation time during significance adjustment.

\subsubsection{Discarding symmetric nodes}
\label{subsubsec:discarding-symmetric-nodes}
In case of equal minimum proportions for all groups $p_1 = p_2 = \ldots = p_|G|$ the mTree shows a convenient property that we can use to reduce additional space and computation time. 
%
Remember that whenever the mcdf value falls below $\alpha$ for a particular position $i$, we have to put a protected candidate onto $i$. 
%
For equal minimum proportions the tree branches into $|G| - 1$ symmetric nodes $m(i)$ of the same likelyhood.
%
As an example reconsider the mTree from Figure~\ref{fig:mtree-symmetric-adjusted} at level 6. 
%
For two protected groups with minimum proportions $[1/3, 1/3]$ we see that the tree branches into two symmetric nodes $[1, 0]$ and $[0,1]$. 
%
Both have the same mcdf values.
%
We store only one of the nodes and flag it as ``has mirrored node'' and continue our mTree computation only in the stored branch.
%
This way we save half of the space and computation time needed, without loosing any information about the tree. 

\subsubsection{Stored mTrees}
\label{subsubsec:stored-mtrees}
During the computation of an adjusted mTree with parameters $k,p_G , \alpha$ we calculate many temporary mTrees (first the unadjusted ones, then the ones for the regression algorithm, then the ones for the binary search steps).
%
We persist all of the temporary mTrees plus the final tree in files for later usage.
%
The filenames contain the tree parameters and whether or not it is adjusted and its probability to fail a fair ranking $\failprob$.
%
If any of these trees is needed at a later point in time it can be loaded from disc instead of being recomputed, be it as input for multinomial \algoFAIR or as temporary tree during a new adjusted mTree computation.





\section{Experiments}\label{sec:experiments}
%\todo{Are we still doing this experiment to verify algorithm correctness?}
%In the first part of our experiments we create synthetic datasets to demonstrate the correctness of the adjustment done by Algorithm \algoCorrect (\S\ref{subsubsec:JuliaExperimentalVerification}).
%
In this section, we consider several public datasets for evaluating the multinomial \algoFAIR algorithm (datasets in \S\ref{sec:experiments-datasets}, metrics and comparison with baselines in \S\ref{sec:experiments-baselines}, and results in \S\ref{sec:experiments-results}).

\begin{table}[t]
	\caption{Datasets and experimental settings.}
	\vspace{-3mm}
	\label{tbl:datasets}
	\resizebox{1.01\columnwidth}{!}{%
		\centering\begin{tabular}{clcclllc}\toprule
			&        &                         &                         & Quality   & Protected & Protected \\
			& Dataset & \multicolumn{1}{c}{$n$} & \multicolumn{1}{c}{$k$} & criterion & groups     & \% \\ 
			\midrule
			D1 & COMPAS \cite{angwin_2016_machine}& 6173 & 500 & ad-hoc score & PoC & 65.9\% \\
			\midrule
			D2 & COMPAS \cite{angwin_2016_machine}& 6173 & 500 & recidivism + & PoC male & 54.7\%\\ 
			& & & & prior arrests + & white male & 26.3\% \\
			& & & & violent recidivism & PoC female & 11.2\% \\
			\midrule
			D3 & COMPAS \cite{angwin_2016_machine}& 6173 & 500 & recidivism + & male, 25yr. < x < 45yr. & 46.1\%\\ 
			& & & & prior arrests + & male, < 25yr. & 17.8\% \\
			& & & & violent recidivism & male, > 45yr. & 17.1\% \\
			& & & & & female, < 25yr. & 4.0\% \\
			& & & & & female, < 25yr. & 3.9\% \\
			\midrule
			D4 & COMPAS \cite{angwin_2016_machine}& 6173 & 500 & recidivism + & PoC, 25yr. < x < 45yr. & 40.0\%\\ 
			& & & & prior arrests + & PoC, < 25yr. & 16.2\% \\
			& & & & violent recidivism & PoC, > 45yr. & 10.8\% \\
			& & & & & White, > 45yr. & 10.2\% \\
			& & & & & White, < 25yr. & 5.6\% \\
			\midrule
			D5 & German credit \cite{lichman_2013_uci} & 1K & 500 & credit rating & female, non-prot. age & 21.7\% \\
			& & & & & male, oldest 10\% & 6.3\% \\
			& & & & & female, youngest 10\% & 6.1\% \\
			& & & & & male, youngest 10\% & 4.4\% \\
			& & & & & female, oldest 10\% & 3.2\% \\
			\midrule
			D6 & LSAT \cite{wightman1998lsac}  & 21K & 2K  & LSAT score  & White, female & 35.3\%  \\ 
			& & & & & PoC, female & 8.4\% \\
			& & & & & PoC, male & 7.6\% \\
			\bottomrule
		\end{tabular}
	}
	\vspace{-3mm}
\end{table}

\subsection{Datasets}\label{sec:experiments-datasets}

Table~\ref{tbl:datasets} summarizes the datasets used in our experiments.
%
Each dataset contains a set of people with demographic attributes, plus a quality attribute.
%
Note that dataset D1 is the same experimental setup as in our previous paper~\cite{zehlike2017fair}, but instead of using the original algorithm \algoFAIR, we run the binomial experiment with the multinomial extension proposed in Section~\ref{sec:problem}.
%
For each dataset, we consider a value of $k$ that is a round number ({\em e.g.}, 500, or 2,000) with $k<n$.
%
For the purposes of these experiments, we considered several scenarios of protected groups.
%
We remark that the choice of protected group is not arbitrary: it is determined completely by law or voluntary commitments; for the purpose of experimentation we test different scenarios, but in a real application there is no ambiguity about which is the protected group and what is the minimum proportion.
%
An experiment consists of generating a ranking using multinomial \algoFAIR and then comparing it with baseline rankings according to the metrics introduced in the next section.
%
We use the three publicly-available datasets: COMPAS~\cite{angwin_2016_machine}, German Credit~\cite{lichman_2013_uci}) and LSAC~\cite{wightman1998lsac}.

\spara{COMPAS} (Correctional Offender Management Profiling for Alternative Sanctions) is an assessment tool for predicting recidivism based on a questionnaire of 137 questions. It is used in several jurisdictions in the US, and has been accused of racial discrimination by producing a higher likelihood to recidivate for African Americans~\cite{angwin_2016_machine}.
%
In our experiment, we test a scenario in which we want to create a fair ranking of the top-$k$ people who are least likely to recidivate, who could be, for instance, considered for a pardon or reduced sentence.
%
We calculated a candidate's overall score as a weighted summation of the columns ``recidivism, violent recidivism'' and ``prior arrests'' from the original dataset.
%
Our protected groups are formed by different combination of the attributes ``race, age'' and ``sex'', where race is either ``white'' or ``non-white'', age is either ``younger than 25'', ``between 25 and 45'' or ``older than 45'', and sex is either ``male'' or ``female''.
%
We observe that non-white people, as well as males are given a larger recidivism score than other groups.
%
We therefore consider white, female and between 25 and 45 as the \emph{non-protected} categories for our experiments.

\spara{German Credit} is the Statlog German Credit Data collected by Hans Hofmann~\cite{lichman_2013_uci}.
%
It is based on credit ratings generated by Schufa, a German private credit agency based on a set of variables for each applicant, including age, gender, marital status, among others. The Schufa score is an essential determinant for every resident in Germany when it comes to evaluating credit rating before getting a phone contract, a long-term apartment rental or almost any loan.
%
We use the credit-worthiness as qualification, calculated as a weighted summation of the features account status, credit duration, credit amount and employment length. 
As protected attributes we use the sex and age of a candidate: females and whether or not they belong to the group of the 100 youngest or oldest persons respectively form the protected groups, because these tend to be given lower scores.

\spara{LSAT} is a dataset collected by~\citet{wightman1998lsac} to study whether the admission metrics to law schools in the US disadvantage students of color. 
%
The qualification attribute consists of scores in the US Law School Admission Test.
%
The protected features are a person's sex and whether or not they belong to the group of people of color (PoC).
%
Women and PoC score on average lower than men and Whites in this dataset, which is why we regard white men as non-protected and the other groups as protected.

\subsection{Baselines and Metrics}\label{sec:experiments-baselines}

For each dataset, we generate various top-$k$ rankings with varying targets of minimum proportion of protected candidates $p$ using \algoFAIR, plus two baseline rankings:

\spara{Baseline 1: Color-blind ranking.} The ranking $c|_k$ that only considers the qualifications of the candidates, without considering group fairness, as described in Section~\ref{concept:color-blind-ranking}.

\spara{Baseline 2: \citet{zehlike2020matching}.} The Continuous Fairness Algorithm (CFA$\theta$) is a post-processing ranking method that aligns the score distributions of the protected candidates with the Wasserstein-barycenter of all group distributions.
%
To achieve this the algorithm finds a new distribution of scoresfor each group, the "fair representation", by interpolating between the barycenter and the group distribution, subject to a given fairness parameter $\theta$.
%
This fair representation corresponds to the idea to rank groups separately and then subsequently pick the best candidates from each group to create the result ranking. 

\spara{Utility (Performance Measure). } We report the loss in ranked utility after score normalization, in which all $q_i$ are normalized to be within $[0, 1]$.
%
We also report the maximum rank drop, {\em i.e.}, the number of positions lost by the candidate that realizes the maximum ordering utility loss.

\spara{NDCG (Performance Measure). }
%
We report a normalized weighted summation of the quality of the elements in the ranking, $\sum_{i=1}^{k} w_i q_{(\tau_i)}$, in which the weights are chosen to have a logarithmic discount in the position:  $w_i = \frac{1}{\log_2 (i+1)}$. This is a standard measure to evaluate search rankings~\cite{jarvelin2002cumulated}.
%
This is normalized so that the maximum value is $1.0$.

\spara{Exposure (Fairness Measure). } We use a measure of average group exposure, which we define as the average position bias that a group is exposed to. 
%
In rankings their exposure to the user is critical for ranked candidates to benefit from the system and if a group of candidates is systematically ranked low, it can be considered as biased~\cite{friedman1996bias}. 
%
We model position bias $v$ by means of a geometric progression of the form $v(k) = \frac{1}{2}^{(k-1)}$. 
%
Thus the first position has a bias $v(1)=1$, which then decreases geometrically.
%
Higher position bias translates into more exposure and hence more visibility.
%
We show that large increases in average group exposure can be achieved with relatively small losses in ordering utility and NDCG.


\subsection{Results}\label{sec:experiments-results}
%
\begin{table}[t]
	\caption{Experimental results, highlighting in boldface the best non-color-blind result. Both FA*IR and the baseline from \citeauthor{Feldman2015} achieve the same target proportion of protected elements in the output and the same selection unfairness, but in general FA*IR achieves it with less ordering unfairness, and with less maximum rank drop (the number of positions that the most unfairly ordered element drops).}
	\vspace{-3mm}
	\label{tbl:results}
	\resizebox{1.01\columnwidth}{!}{%
		\centering\begin{tabular}{llcccccc}\toprule
			&        & \% Prot. &       & Ordering     & Rank & Selection \\
			& Method & output   & NDCG  & utility loss & drop & utility loss \\ \midrule
			D1 (51.2\%) & Color-blind & 25\% & 1.0000 & 0.0000 & 0 & 0.0000 \\
			COMPAS, & FA*IR p=0.5 & 46\% & \textbf{0.9858} & \textbf{0.2026} & \textbf{319} & \textbf{0.1087} \\
			race$=$Afr.-Am. & \citeauthor{Feldman2015} & 51\% & 0.9779 & 0.2281 & 393 & 0.1301 \\ \midrule
			
			D2 (80.7\%) & Color-blind & 73\% & 1.0000 & 0.0000 & 0 & 0.0000 \\
			COMPAS, & FA*IR p=0.8 & 77\% & \textbf{1.0000} & \textbf{0.1194} & \textbf{161} & \textbf{0.0320} \\
			gender$=$male & \citeauthor{Feldman2015} & 81\% & 0.9973 & 0.2090 & 294 & 0.0533 \\ \midrule
			
			D3 (19.3\%) & Color-blind & 28\% & 1.0000 & 0.0000 & 0 & 0.0000 \\
			COMPAS, & FA*IR p=0.2 & 28\% & \textbf{0.9999} & \textbf{0.2239} & \textbf{1} & \textbf{0.0000} \\
			gender$=$female & \citeauthor{Feldman2015} & 19\% & 0.9972 & 0.3028 & 278 & 0.0533 \\ \midrule
			
			D4 (69.0\%) & Color-blind & 74\% & 1.0000 & 0.0000 & 0 & 0.0000 \\
			Ger. cred, & FA*IR p=0.7 & 74\% & \textbf{1.0000} & \textbf{0.0000} & \textbf{0} & \textbf{0.0000} \\
			gender$=$female & \citeauthor{Feldman2015} & 69\% & 0.9988 & 0.1197 & 8 & 0.0224 \\ \midrule
			
			D5 (14.9\%) & Color-blind & 9\% & 1.0000 & 0.0000 & 0 & 0.0000 \\
			Ger. cred,& FA*IR p=0.2 & 15\% & \textbf{0.9983} & \textbf{0.0436} & \textbf{7} & \textbf{0.0462} \\
			age $<$ 25 & \citeauthor{Feldman2015} & 15\% & 0.9952 & 0.1656 & 8 & \textbf{0.0462} \\ \midrule
			
			D6 (54.8\%) & Color-blind & 24\% & 1.0000 & 0.0000 & 0 & 0.0000 \\
			Ger. cred, & FA*IR p=0.6 & 50\% & \textbf{0.9913} & \textbf{0.1137} & \textbf{30} & \textbf{0.0593} \\
			age $<$ 35 & \citeauthor{Feldman2015} & 55\% & 0.9853 & 0.2123 & 36 & 0.0633 \\ \midrule
			
			D7 (53.1\%) & Color-blind    & 49\% & 1.0000 & 0.0000 & 0 & 0.0000 \\
			SAT, 	    & FA*IR p=0.6    & 57\% & \textbf{0.9996} & \textbf{0.0167} & 365 & 0.0083 \\ 
			gender$=$female & \citeauthor{Feldman2015} & 56\% & \textbf{0.9996} & \textbf{0.0167} & \textbf{241} & \textbf{0.0042} \\ \midrule
			
			D8a (27.5\%)  & Color-blind    & 28\% & 1.0000 		& 0.0000 	& 0  & 0.0000 \\
			Economist,     & FA*IR p=0.3    & 28\% & \textbf{1.0000} & \textbf{0.0000} & \textbf{0}  & \textbf{0.0000} \\
			gender$=$female & \citeauthor{Feldman2015} & 28\% & 0.9935 		& 0.6109 	& 5 & \textbf{0.0000} \\ \midrule
			D8b (42.5\%)  & Color-blind    & 43\% & 1.0000 		& 0.0000 	& 0           & 0.0000 \\
			Mkt. Analyst,  & FA*IR p=0.4    & 43\% & \textbf{1.0000} & \textbf{0.0000} & \textbf{0}  & \textbf{0.0000} \\
			gender$=$male & \citeauthor{Feldman2015} & 43\% & 0.9422 		& 1.0000 	& 5 & \textbf{0.0000} \\ \midrule
			D8c (29.7\%)  & Color-blind    & 30\% & 1.0000 		& 0.0000 	& 0           & 0.0000 \\
			Copywriter,    & FA*IR p=0.3    & 30\% & \textbf{1.0000} & \textbf{0.0000} & \textbf{0}  & \textbf{0.0000} \\
			gender$=$female & \citeauthor{Feldman2015} & 30\% & 0.9782 & 0.4468 & 10 & \textbf{0.0000} \\
			\bottomrule
		\end{tabular}
	}
	\vspace{-3mm}
\end{table}
%
Table~\ref{tbl:results} summarizes the results. We report on the result using $p_G$ in two different settings, namely as a statistical parity vector i.e. the $p$-values for each group correspond to their respective proportions in the dataset and as a vector with all $p$-values being equal. 
%
First, we observe that in general changes in utility with respect to the color-blind ranking are minor, as the utility is dominated by the top positions, which do not change dramatically.
%
\meike{Revise this: Second, \algoFAIR achieves higher or equal selection utility than the baseline~\cite{zehlike2020matching} in all but one of the experimental conditions (D7).
%
Third, \algoFAIR achieves higher or equal ordering utility in all conditions. This is also reflected in the rank loss of the most unfairly
treated candidate included in the ranking ({\em i.e.}, the candidate that achieves the maximum ordering utility loss).} %, which is always equal or less for \algoFAIR.

Importantly, \algoFAIR allows to create rankings for user-defined values of $p$ and in particular for values beyond statistical parity, something that cannot be done directly with the baseline (\citet{zehlike2020matching} allows at maximum statistical parity in the result ranking when setting $\theta=1$, i.e. to the maximum value).
%
\meike{see how to plot a figure from the results: Figure~\ref{fig:results-moving-p} shows results when varying $p$ in dataset D4 (German credit, the protected group is people under 25 years old).
%
This means that \algoFAIR allows a wide range of positive actions, for instance, offering favorable credit conditions to people with good credit rating, with a preference towards younger customers.
%
In this case, the figure shows that we can double the proportion of young people in the top-$k$ ranking (from the original 15\% up to 30\%) without introducing a large ordering utility loss and maintaining NDCG almost unchanged.}

\begin{figure}[t!]
	\centering
	\subfloat[Ordering utility]{\includegraphics[width=.49\columnwidth]{pics/d4-protected-vs-ordering.png}}
	\subfloat[NDCG]{\includegraphics[width=.49\columnwidth]{pics/d4-protected-vs-ndcg.png}}
	\vspace{-2mm}
	\caption{\meike{replace figure} Depiction of possible trade-offs using \algoFAIR. Increase in the percentage of protected candidates in D5 (German credit, protected group age $<$ 25) for increasing values of $p$, compared to %decrease in ordering utility (left) and decrease in NDCG (right).}
		ordering utility and NDCG.}
	\vspace{-\baselineskip}
	\label{fig:results-moving-p}
\end{figure}

\section{Corrections for the Significance Adjustment in~\cite{zehlike2017fair}. }
\label{sec:adjustment-binomial}
%
In this section we describe extensions for the significance adjustment from~\cite{zehlike2017fair}.
%
We also provide a correction of the procedure which did not work for very small $k$ and $\alpha$. 

For binomial distributions, i.e. where only one protected and one non-protected group is present, the inverse CDF can be stored as a simple table, which we compute using Algorithm~\ref{alg:constructMTable}. 
%
Analogous to the multinomial case, we will call such a table \textit{mTable}.
%
Table~\ref{tbl:ranked_group_fairness_table} shows an example of such a pre-computed table with different $ k $ and $ p $, using $\alpha=0.1$.
%
For instance, for $p=0.5$ we see that at least 1 candidate from the protected group is needed in the top 4 positions, and 2 protected candidates in the top 7 positions.
\begin{algorithm}[h]
	\caption{Algorithm \algoMtable computes the data structure to efficiently verify or construct a ranking that satisfies binomial ranked group fairness.}
	\label{alg:constructMTable}
	\small
	\AlgInput{$k$, the size of the ranking to produce; $p$, the expected proportion of protected elements; $\alphaadj$, the significance for each individual test.}
	\AlgOutput{$ \mtable $: A list that contains the minimum number of protected candidates required at each position of a ranking of size $k$.}
	$\mtable \leftarrow [k]$ \AlgComment{list of size $k$} 
	\For{$i \leftarrow 1$ \KwTo $k$}{
		$\mtable[i] \leftarrow F^{-1}(i,p,\alphaadj)$ \AlgComment{the inverse binomial cdf}
	}
	\Return{$ \mtable $ }
\end{algorithm}

\begin{table}[h!]
	\small\begin{tabular}{r|cccccccccccc}
		\diaghead{some text}%
		{p}{k}&
		% & \multicolumn{10}{c}{k} \\
		1 & 2 & 3 & 4 & 5 & 6 & 7 & 8 & 9 & 10 & 11 & 12 \\ \midrule
		0.1      & 0 & 0 & 0 & 0 & 0 & 0 & 0 & 0 & 0 & 0  &  0 &  0 \\
		0.2      & 0 & 0 & 0 & 0 & 0 & 0 & 0 & 0 & 0 & 0  &  1 &  1 \\
		0.3      & 0 & 0 & 0 & 0 & 0 & 0 & 1 & 1 & 1 & 1  &  1 &  2 \\
		0.4      & 0 & 0 & 0 & 0 & 1 & 1 & 1 & 1 & 2 & 2  &  2 &  3 \\
		0.5      & 0 & 0 & 0 & 1 & 1 & 1 & 2 & 2 & 3 & 3  &  3 &  4 \\
		0.6      & 0 & 0 & 1 & 1 & 2 & 2 & 3 & 3 & 4 & 4  &  5 &  5 \\
		0.7      & 0 & 1 & 1 & 2 & 2 & 3 & 3 & 4 & 5 & 5  &  6 &  6 \\
		\bottomrule
	\end{tabular}
	\caption{Example values of $m_{\alpha,p}(k)$, the minimum number of candidates in the protected group that must appear in the top $k$ positions to pass the ranked group fairness criteria with $\alpha=0.1$ in a binomial setting.}
	\label{tbl:ranked_group_fairness_table}
\end{table}

Figure~\ref{fig:why-adjustment-is-needed-binomial} shows that we need a correction for $\alpha$ also in the binomial case (note that the scale is logarithmic).
%
In the following, we show that the special case of having only one protected group offers new possibilities for verifying ranked group fairness. 
%
A key improvement is that in the binomial setting we can calculate the exact failure probability $\failprob$ (i.e. a fair ranking gets rejected by the ranked group fairness test), which results in a much more efficient binary search for $\alphaadj$. 
%
Remember that in we had to estimate this probability for mTrees (see Section~\ref{sec:model-adjustment}) using an experimental procedure. 
%
First we introduce the necessary notation for the binomial case and describe how we calculate the exact $\failprob$.
%
Then we show that we can divide the continuum of possible $\alpha$ values in discrete parts in order to be able to apply efficient binary search for the most accurate $\alphaadj$.
%
Last we analyze the complexity of the proposed algorithms.
%This figure is generated by simulation, generating rankings using the process described above and showing the probability of those rankings being rejected by our ranked group fairness test with $\alphaadj=0.1$.
%
%The figure suggests that depending on $k$ we would need to change the value of $\alpha$ if we want to achieve a rejection rate of $0.1$.


\begin{figure}[h]
	\centering
	{\includegraphics[width=.48\textwidth]{pics/failProbPlotBinom.png}}
	\caption{
		Probability that a fair ranking created by a Bernoulli process with $p=0.5$ fails the ranked group fairness test.\label{fig:why-adjustment-is-needed-binomial}
		%
		Experiments on data generated by a simulation, showing the need for multiple tests correction.
		%
		The data has: one protected group, with a ranking created by  a Bernoulli process (Fig.~\ref{fig:why-adjustment-is-needed-binomial}); and two protected groups, with a ranking created by a multinomial process (Fig.~\ref{fig:why-adjustment-is-needed-multinomial}).
		%
		Rankings should have been rejected as unfair at a rate $\alpha = 0.1$.
		%
		However, we see that the rejection probability increases with $k$.
		%
		Note the scale of $k$ is logarithmic.}
	\label{fig:need-for-model-adjustment}
\end{figure}

\subsection{Success Probability for One Protected Group}\label{subsubsec:adjustment-binomial}

The probability $\successprob$ that a fair ranking, created by the process of~\citet{yang2016measuring}, passes the ranked group fairness test (i.e. the success probability) with parameters $(p,\alpha)$ can be computed as follows:
%
Let $m(k) = m_{\alpha,p}(k) = F^{-1}(k,p,\alpha)$ be the number of protected elements required up to position $k$.
%
Let $\minv(i) = k$ s.t. $m(k) = i$ be the position at which $i$ protected elements are required.
%
Let $b(i) = \minv(i) - \minv(i-1)$ (with $\minv(0) = 0$) be the size of a ``block,'' that is, the gap between one increase and the next in $m(\cdot)$.
%
We call the $k$-dimensional vector $(m(1), m(2), \ldots , m(k))$ a \emph{mTable}.
%
An example is shown on Table~\ref{tbl:05:example_blocks}.
%
\begin{table}[b!]
	\centering
	\begin{tabular}{cccccccccccccc}\toprule
		$k$    & 1 & 2 & 3 & \textbf{{4}} & 5 & 6 & \textbf{7} & 8 & \textbf{9} & 10 & 11 & \textbf{12} \\
		\midrule
		$m(k)$ & 0 & 0 & 0 & \multicolumn{1}{c|}{1} & 1 & 1 & \multicolumn{1}{c|}{2} & 2 & \multicolumn{1}{c|}{3} & 3  & 3  & \multicolumn{1}{c}{4}\\
		Inverse   & \multicolumn{4}{c|}{$\minv(1)=4$}
		& \multicolumn{3}{c|}{$\minv(2)=7$}
		& \multicolumn{2}{c|}{$\minv(3)=9$}
		& \multicolumn{3}{c}{$\minv(4)=12$}\\
		Blocks       & \multicolumn{4}{c|}{$b(1)=4$}
		& \multicolumn{3}{c|}{$b(2)=3$}
		& \multicolumn{2}{c|}{$b(3)=2$}
		& \multicolumn{3}{c}{$b(4)=3$}\\
		\bottomrule
	\end{tabular}
	\caption[Example of different block sizes]{Example of $m(\cdot)$, $\minv(\cdot)$, and $b(\cdot)$ for $p=0.5, \alpha=0.1$.}
	\label{tbl:05:example_blocks}
\end{table}

\noindent Furthermore let
\begin{equation}
	\label{eq:05:combinations}
	I_{m(k)} = \{ v = (i_1, i_2, \ldots, i_{m(k)}): \forall \ell' \in \lbrace 1,\ldots,m(k) -1 \rbrace, 0 \le i_{\ell'} \le b(\ell') \wedge \sum_{j=1}^{\ell'} i_j \ge \ell' \}
\end{equation}
%
represent all possible ways in which a fair ranking\footnote{Note that we do not consider rankings of size 0, which always pass the test.} generated by the method of \citet{yang2016measuring} can pass the ranked group fairness test, with $i_j$ corresponding to the number of protected elements in block $j \; (\text{with } 1 \le j \le k)$.
%
As an example consider again Table~\ref{tbl:05:example_blocks}: the first block contains four positions, i.e. $b(1)=4$ and this block passes the ranked group fairness test, if it contains at least one protected candidate, hence $i_1 \in \{1, 2, 3, 4\}$.
%
The probability of considering a ranking of $k$ elements (i.e. $m(k)$ blocks) unfair, is:
\begin{equation}
	\label{eq:05:failureProb}
	\failprob = 1 - \successprob = 1 - \sum_{v \in I_{m(k)}} \prod_{j=1}^{m(k)} f(v_j; b(j), p)
\end{equation}
%
\noindent where $f(x;b(j),p) = Pr(X = x)$ is the probability density function (PDF) of a binomially distributed variable $X \sim Bin(b(j), p)$.
%
However, if calculated naively this expression is intractable because of the large number of combinations in $I_{m(k)}$.
%

\setlength{\textfloatsep}{2pt}% Remove \textfloatsep
\begin{algorithm}[t!]
	\caption{Algorithm \algoRecursive computes the probability, that a given mTable accepts a fair ranking (see right term of Eq.~\ref{eq:05:failureProb}).}
	\label{alg:05:successProb} % But whenever possible refer to this algo. by name not number
	\small
	\AlgInput{
		$\texttt{b[]}$ list of block lengths (Table~\ref{tbl:05:example_blocks}, 3rd line);\\ 
		$\texttt{maxProtected}$ the sum of all entries of $\texttt{b[]}$;\\
		$\texttt{currentBlockIndex}$ index of the current block; \\
		$\texttt{candidatesAssigned}$ number of protected candidates assigned for the current possible solution; \\
		$p$, the expected proportion of protected elements.}
	\AlgOutput{The probability of accepting a fair ranking.}
	
	\If{$\texttt{b[].length} = 0$}{
		\Return{$1$}
	}
	\tcp{we need to assign at least one protected candidate to each block}
	$\texttt{minNeededThisBlock} \leftarrow \texttt{currentBlockIndex} - \texttt{candidatesAssigned}$\\
	\tcp{if we already assigned enough candidates, minNeededThisBlock = 0 (termination condition for the recursion)}
	\If{$\texttt{minNeededThisBlock} < 0$}{
		$\texttt{minNeededThisBlock} \leftarrow 0$
	}
	$\texttt{maxPossibleThisBlock} \leftarrow \textit{argmin}(\texttt{b[0]}, \texttt{maxProtected})$ \\
	$\texttt{assignments} \leftarrow 0$ \\
	$\texttt{successProb} \leftarrow 0$ \\
	\tcp{sublist without the first entry of $\texttt{b[]}$}
	$\texttt{b\_new[]} \leftarrow \textit{sublist}(\texttt{b[]}, 1, \texttt{b[].length})$ \label{algoline:05:suffixes}\\
	$\texttt{itemsThisBlock} \leftarrow \texttt{minNeededThisBlock}$\\
	\While{$\texttt{itemsThisBlock} \leq \texttt{maxPossibleThisBlock}$}{
		$\texttt{remainingCandidates} \leftarrow \texttt{maxProtected} - \texttt{itemsThisBlock}$ \\
		$\texttt{candidatesAssigned} \leftarrow \texttt{candidatesAssigned} + \texttt{itemsThisBlock}$ \\
		\tcp{each recursion returns the success probability of \emph{all possible ways} to fairly rank protected candidates after this block}
		$\texttt{suffixSuccessProb} \leftarrow \textsc{\algoRecursive} ( $ \\ \pushline $ \texttt{remainingCandidates},\texttt{b\_new[]}, \texttt{currentBlockIndex} + 1,$ \\ $ \texttt{candidatesAssigned})$ 
		\label{algoline:05:recursion}\\
		\popline $\texttt{totalSuccessProb} \leftarrow \texttt{totalSuccessProb} \; + $ \\ \pushline $ \textsc{PDF}(\texttt{maxPossibleThisBlock}, \texttt{itemsThisBlock}, p) \; \cdot $ \\ $ \texttt{suffixSuccessProb}$ \label{algoline:05:pdf}\\
		\popline $\texttt{itemsThisBlock} \leftarrow \texttt{itemsThisBlock} + 1$\\
	}
	\Return{probability of accepting a fair ranking: $\texttt{totalSuccessProb}$ }
\end{algorithm}
We therefore propose a dynamic programming method, Algorithm~\ref{alg:05:successProb}, which computes the probability that a fair ranking passes the ranked group fairness test (i.e. the right term of Equation~\ref{eq:05:failureProb}) \emph{recursively}.
%
Note that because of the combinatorial complexity of the problem a \emph{simple closed-form expression} to compute $\failprob$ is unlikely to exist.
%With this algorithm however, we have to account for an important pitfall: we may choose a combination of $k, p, \alpha$ where the last block ``reaches beyond'' $k$.
%
%Consider again Table~\ref{tbl:05:example_blocks} with $k=8, \minpro$p=$0.5, \alpha=0.1$: the second block reaches until position 7, while the third block reaches until position 9.
%
%If we cut off the mTable at $k=8$ to adjust the significance level, the last block reaches beyond $k$ which is to position 9.
%
%\meike{Inwiefern macht der neue Algorithmus das Problem mit aufgebrochenen Blöcken besser?}
%To account for that t
The algorithm breaks the vector $v = (i_1, i_2, \ldots ,i_{\ell})$ of Equation~\ref{eq:05:combinations} into a \textit{prefix} and a \textit{suffix} (Alg.~\ref{alg:05:successProb}, Line~\ref{algoline:05:suffixes}).
%
We call $i_1$ the \textit{prefix} of $(i_2, \ldots, i_{\ell})$, and $(i_2, \ldots, i_{\ell})$ the \textit{suffix} of $i_1$.
%
The algorithm starts with a prefix and calculates all possible suffixes, that pass the ranked group fairness test, recursively (Line~\ref{algoline:05:recursion}).
%
Consider the following example: for the first prefix $i_1 = 1$ the algorithm computes all possible suffixes, where we rank exactly one protected candidate in the first block.
%
For this prefix $i_1$, combined with each possible suffix $v \setminus i_1$, we calculate the success probability $\prod_{j=1}^{m(k)} f(v_j; b(j), p)$ for each instance of $v$ (Line~\ref{algoline:05:pdf}).
%
In the next recursion level we start with a new prefix, let us say $i_1=1, i_2 =1$.
%
The algorithm computes all possible suffixes, i.e. all rankings where we rank exactly one protected candidates in the first block and one protected candidate in the second block.
%
Then it computes the respective success probabilities.
%
This procedure continues for $m(k)$ iterations.
%
After that the whole program starts again with $i_1=2$ and is repeated until the maximum number of protected candidates is reached, in our case $b(1)=4$.
%
All intermediate success probabilities are added up (Line~\ref{algoline:05:pdf}) to the total success probability (see Eq.~\ref{eq:05:failureProb}) of the mTable that was created given $k, p, \alpha$.

Note that there are at most $\prod_{j=1}^{m(k)}b(j)$ possible combinations to distribute the protected candidates within the blocks.
%
Furthermore many $v \in I_{m(k)}$ share the same prefix and hence have the same probability density value for these prefixes.
%
To reduce computation time the algorithm stores the binomial probability density value for each prefix in a hash map with the prefix as key and the respective pdf as value.
%
Thus the overall computational complexity becomes $O(\prod_{j=1}^{m(k)}b(j) \cdot O(\texttt{binomPDF}))$.

\subsection{Finding the Correct mTable}\label{subsec:finding-mtable}
To find the correct mTable, that is the mTable with an overall success probability of $\successprob = 1-\alpha$, we propose Algorithm~\ref{alg:05:binarySearch} that takes  parameters $k, p, \alphaadj$ as input.
%
%We use Algorithm~\ref{alg:05:successProb} to determine the adjusted significance level $\alphaadj$ for the ranked group fairness test, such that the overall acceptance probability for the mTable with parameters $k, p, \alphaadj$ becomes $1-\alpha$.
%
It sequentially creates mTables (recall that these are $k$-dimensional vectors of the form $(m(1), m(2), \ldots , m(k))$) for different values of $\alphaadj$, and then calls Algorithm~\ref{alg:05:successProb} to calculate their success probability until it finds the correct mTable with overall failure probability $\failprob = \alpha$ . 
%
Our goal is to use binary search to select possible candidates for $\alphaadj$ systematically.

%\note[ChaTo]{Perhaps the definition of the mass of an mTable can wait a few more paragraphs, until you need it.}
However, to be able to do binary search, we need a discrete measure for the $\alpha$-space to search on, otherwise the search would never stop.
%
Furthermore, to reduce complexity we only want to consider mTables with certain properties, which we define in the following paragraph. 
%
A $k$-dimensional vector (e.g. $(0,0,1,2,3)$) has to have two properties in order to constitute a mTable, rather than just a vector of natural numbers: it has to be \emph{valid} and \emph{legal}.
%
\begin{definition}[Valid mTable]
	\label{def:05:valid-mtable}
	The $\text{mTable}_{p,k,\alpha}=(m(1) , m(2) , \ldots , m(k))$ is \emph{valid} if and only if, $m(i) \leq m(j)$ for all $i,j \in \lbrace 0, \ldots, k \rbrace$ with $i < j$ and $m(i)=n \Rightarrow m(i+1) \leq n+1$.
\end{definition}
\noindent It is easy to see that many valid mTables exist.
%
They correspond to all $k$-dimensional arrays with integers monotonically increasing by array indices.
%
However we only want to consider those valid mTables for our ranked group fairness test that have been created by the statistical process in~\citet{yang2016measuring}.
%
We call these \textit{legal mTables}.
%
\begin{definition}[Legal mTable]
	\label{def:05:legal-mtable}
	A $\text{mTable}_{p,\alpha,k}$ is \textit{legal} if and only if there exists a $p,k,\alpha$ such that
	$\texttt{constructMTable}(p,k,\alpha)=\text{mTable}_{p,\alpha,k}$.
\end{definition}
%
\meike{@Tom: add an explanation paragraph here.}
\begin{definition}[constructMTable]
	\label{def:05:construct-mtable-single-test}
	For $p\in [0,1], k \in \mathbb{N}, \alpha \in [0,1]$ we define a function to construct a mTable from input parameters $p, k, \alpha$ according to \cite{yang2016measuring}.
	
	\noindent$\texttt{constructMTable} : \\ (0,1) \times \mathbb{N} \times [0,1] \longrightarrow \lbrace (m(1) ,\ldots, m(k)): m(i) = F^{-1}(i,p,\alpha), \, i = \{1,\ldots,k\}\rbrace$ \\
	with $\texttt{constructMTable}(p,k,\alpha)=\text{mTable}_{p,k,\alpha}$.
\end{definition}
%
\begin{lemma}
	\label{lemma:05:legal-valid-mtable}
	If a mTable is legal, it is also valid.
\end{lemma}
%
\noindent Lemma \ref{lemma:05:legal-valid-mtable} follows directly by construction.
%
Now we need a discrete partition of the continuous $\alpha$ space, that is a discrete measure that corresponds to exactly one legal mTable for a given set of parameters $k,p,\alpha$. 
%
We call this measure the \emph{mass} of a mTable.
%The definition of a legal mTable together with its corresponding mass enables us to perform a binary search, which helps us to find $\alphaadj$ and hence the correct mTable with an overall failure probability of $\alpha$.
%
%Note that for any value of $\alpha$ there exists exactly one \emph{legal} mTable.
%
%However, the opposite is not true: as $\alpha$ is a real number from the interval $[0, 1]$, the same mTable can be created using different (but very similar) values of $\alpha$.
%
%Thus without a discrete measure for a mTable we do not know when to stop searching, which is why we introduce the \emph{mass of a mTable}.
%
\begin{definition}[Mass of a mTable]
	\label{def:05:Mass of a MTable}
	For $\text{mTable}_{p,k,\alpha}=(m(1) , m(2) , \ldots , m(k))$ we call\\
	$L_1(\text{mTable}_{p,k,\alpha})=\sum_{i=1}^k m(i)$ the \textit{mass of $\text{mTable}_{p,\alpha,k}$}.
\end{definition}
%
In the following we relate the continuous $\alpha$-space to the discrete mass of a mTable.
%
\begin{lemma}
	\label{lemma:05:non-decreasing-with-alpha-mtable}
	Every $\text{mTable}_{p,k,\alpha}=\texttt{constructMTable}(p,k,\alpha)=(m(1) , m(2) , \ldots , m(k))$ is non-decreasing with $\alpha$. This means that
	$\texttt{constructMTable}(p,k,\alpha - \epsilon) = (m(1)' , m(2)' , \ldots , m(k)')$ will result in $m(i)' \leq m(i)$ for $i=1,\ldots,k$ and $\epsilon > 0$.
\end{lemma}
\begin{proof}
\label{proof:05:non-decreasing-with-alpha-mtable}
Every entry $m(i)$ for $i=1,\ldots,k$ is computed by line 3 of Algorithm~\ref{alg:constructMTable}, i.e. every entry is the inverse binomial cdf $F^{-1}(i,p,\alphaadj)$. In other words $m(i)$ is the smallest integer such that\\
\begin{equation}
\alphaadj \leq \sum_{j=0}^{m(i)}\binom{i}{j}p^i (1-p)^{i-j} = F^{-1}(i,p,\alphaadj)
\end{equation}
The following equivalence holds:
\begin{equation}\label{eq:smaller m}
\alphaadj - \epsilon \leq \sum_{j=0}^{m'(i)}\binom{i}{j}p^i (1-p)^{i-j} = F^{-1}(i,p,\alphaadj-\epsilon)
	\Leftrightarrow \alphaadj \leq \sum_{j=0}^{m'(i)}\binom{i}{j}p^i (1-p)^{i-j} + \epsilon
\end{equation}
Now suppose that $m'(i) > m(i)$ which contradicts lemma \ref{lemma:05:non-decreasing-with-alpha-mtable}. 
%
Then it is that
\[m'(i)>m(i) \Rightarrow \sum_{j=0}^{m'(i)}\binom{i}{j}p^i (1-p)^{i-j} \geq \sum_{j=0}^{m(i)}\binom{i}{j}p^i (1-p)^{i-j}\]
because $\binom{i}{j}p^i (1-p)^{i-j} \geq 0 \; \forall i,j,p$. It follows for $\epsilon >0$ that
\begin{equation}
\alphaadj \leq \sum_{j=0}^{m(i)}\binom{i}{j}p^i (1-p)^{i-j} + \epsilon \leq \sum_{j=0}^{m'(i)}\binom{i}{j}p^i (1-p)^{i-j} + \epsilon
\end{equation}
But then $m'(i)\neq F^{-1}(i,p,\alphaadj-\epsilon)$ because $m(i)$ would be the smaller integer that satisfies Equation~\ref{eq:smaller m}. Thus it has to be that $m'(i)\leq m(i)$.
\end{proof}
%
\noindent This property shows that, if we reduce $\alpha$ in our binary search, the mass of the corresponding mTable is also reduced or stays the same.
%
It very usefully implies a criterion to stop the binary search: namely we stop the calculation when the mass of the mTable at the left search boundary equals the right search boundary.
%
Of course this only works if there exists exactly one legal mTable for each mass, which we proof in the following.
\begin{theorem}
	\label{theorem:05:mtable-mass-injection}
	For fix $p,k$ there exists exactly one legal mTable for each mass $L_1\in \lbrace 1,\ldots,k \rbrace$.
\end{theorem}
%
\begin{proof}
	\label{proof:05:mtable-mass-injection}
	We prove this by contradiction: Let $MT_{p,k,\alpha_1}$ and $MT'_{p,k, \alpha_2}$ be two different mTables with $L_1(MT_{p,k,\alpha_1 }) = L_1(MT'_{p,k,\alpha_2})$.\\
	%
	If both are legal then it applies that $\texttt{constructMTable}(p,\alpha_1 ,k)=MT_{p,\alpha_1 ,k}$
	and \\ $\texttt{constructMTable}(p,\alpha_2 ,k)=MT'_{p,\alpha_2 ,k}$.
	%
	Because $MT_{p,k,\alpha_1} \neq MT'_{p,k,\alpha_2}$, without loss of generality entries $m(i), m(i)' , m(j) , m(j)'$ exist in each table, such that $|m(i) - m(i)'| = |m(j) - m(j)'|$ while at the same time $m(i) > m(i)'$ , $m(j) < m(j)'$ for $i<j, i,j \in \lbrace 1, \ldots , k \rbrace$.
	%
	(Think of it as the two entries in each table "evening out", such that both tables have the same mass.)\\
	%
	If $\alpha_1 > \alpha_2$, then the statement $m(j)' > m(j)$ violates Lemma~\ref{lemma:05:non-decreasing-with-alpha-mtable}.
	%
	If $\alpha_2 > \alpha_1$, then the statement $m(i) > m(i)'$ also violates Lemma~\ref{lemma:05:non-decreasing-with-alpha-mtable}.
	%
	The only possibility left is hence that $\alpha_1 = \alpha_2$, which contradicts $MT \neq MT'$, as both are created using function \texttt{constructMTable}.
\end{proof}
%
With these mathematical properties we can perform a binary search on the continuous $\alpha$-space to find the corrected significance level $\alphaadj$.
%
This corrected significance is used to compute a final mTable with an overall failure probability $\failprob = \alpha$.
\begin{algorithm}[t!]
	\caption{Algorithm \algoBinomBinary calculates the corrected significance level $\alpha_c$ and the mTable $m_{\alpha_c , k, p)}$ with an overall probability $\alpha$ of rejecting a fair ranking.}
	\label{alg:05:binarySearch} % But whenever possible refer to this algo. by name not number
	\footnotesize
	\AlgInput{$k$, the size of the ranking to produce; $p$, the expected proportion of protected elements; $\alpha$, the desired significance level.}
	\AlgOutput{$\alphaadj$ the adjusted significance level; \texttt{m\_{adjusted}} the adjusted mTable}
	\AlgComment{initialize all needed variables}
	\texttt{aMin $\leftarrow$ 0};
	\texttt{aMax $\leftarrow \alpha$ };
	\texttt{aMid} $\leftarrow \frac{(\texttt{aMin + aMax})}{2}$ \\
	\texttt{m\_min} $\leftarrow$ \texttt{constructMTable(k,p,aMin)}; 
	\texttt{m\_max} $\leftarrow$ \texttt{constructMTable(k,p,aMax)}; \\
	\texttt{m\_mid} $\leftarrow$ \texttt{constructMTable(k,p,aMid)} \\
	\texttt{maxMass} $\leftarrow$ \texttt{m\_max.getMass()};
	\texttt{minMass} $\leftarrow$ \texttt{m\_min.getMass()};
	\texttt{midMass} $\leftarrow$ \texttt{m\_mid.getMass()}\\
	
	\While{\texttt{minMass} $<$ \texttt{maxMass} AND \texttt{m\_mid.getFailProb()} $\neq \alpha$ }{
		\If{\texttt{m\_mid.getFailProb()} $< \alpha$}{
			\texttt{aMin} $\leftarrow$ \texttt{aMid}
			\texttt{m\_min} $\leftarrow$ \texttt{constructMTable(k,p,aMin)} \\
		}
		\If{\texttt{m\_mid.getFailProb()} $> \alpha$}{
			\texttt{aMax} $\leftarrow$ \texttt{aMid} \\
			\texttt{m\_max} $\leftarrow$ \texttt{constructMTable(k,p,aMax)} \\
		}
		\texttt{aMid} $\leftarrow \frac{(\texttt{aMin + aMax})}{2}$ \\
		\tcp{stop criteria if midMass equals maxMass or midMass equals minMass}
		\If{\texttt{maxMass - minMass == 1}}{
			\texttt{minDiff} $\leftarrow |$\texttt{m\_min.getFailProb() - }$\alpha|$ \\
			\texttt{maxDiff} $\leftarrow |$\texttt{m\_max.getFailProb() - }$\alpha|$ \\
			\tcp{return the $\alpha_c$ which has the lowest difference from the desired significance}			
			\If{\texttt{minDiff} $<$ \texttt{maxDiff}}{
				\Return{\texttt{aMin, m\_min}}
			}
			\Else{
				\Return{\texttt{aMax, m\_max}}
			}
		}
		\tcp{stop criteria if midMaxx is exactly the mass between minMass and maxMass}
		\If{\texttt{maxMass - midMass == 1} AND \texttt{midMass - minMass == 1}}{
			\texttt{minDiff} $\leftarrow |$\texttt{m\_min.getFailProb() - }$\alpha|$ \\
			\texttt{maxDiff} $\leftarrow |$\texttt{m\_max.getFailProb() - }$\alpha|$ \\
			\texttt{midDiff} $\leftarrow |$\texttt{m\_mid.getFailProb() - }$\alpha|$ \\
			\tcp{return the $\alpha_c$ which has the lowest difference from the desired significance}			
			\If{\texttt{midDiff} $\leq$ \texttt{maxDiff} AND \texttt{midDiff} $\leq$ \texttt{minDiff}}{
				\Return{\texttt{aMid, m\_mid}}
			}
			\If{\texttt{minDiff} $\leq$ \texttt{midDiff} AND \texttt{minDiff} $\leq$ \texttt{maxDiff}}{
				\Return{\texttt{aMin, m\_min}}
			}
			\Else{
				\Return{\texttt{aMax, m\_max}}
			}
		}
	}
	\Return{\texttt{aMid, m\_mid}}
\end{algorithm}

\subsection{Complexity in the Binomial Case} 
\begin{table}[t!]
	\caption{Time complexity for all algorithms for one protected group without pre-computed results.\label{tbl:time-space-binom}}
	\vspace{-4mm}
	\scalebox{0.75}{
		\begin{tabular}{lll}
			\toprule
			\textbf{Algorithm} & \textbf{Time Complexity} & \textbf{Space Complexity}\\
			\midrule
			\rowcolor[HTML]{C0C0C0}
			\algoMtable & $\mathcal{O}(k) \cdot \mathcal{O}(F^{-1}(p,k,\alpha))$ & $\mathcal{O}(k)$ \\
			\algoRecursive & $\mathcal{O}(\texttt{\algoMtable}) + \mathcal{O}(\prod_{j=1}^{m(k)}b(j) \cdot O(\texttt{binomPDF}))$ & $\mathcal{O}(k)$ \\
			\rowcolor[HTML]{C0C0C0}
			\algoBinomBinary & $\mathcal{O}(\log{}k^2) \cdot (\mathcal{O}(\texttt{\algoRecursive}))$ & $\mathcal{O}(k)$  \\
			\bottomrule
		\end{tabular}
	}
\end{table}

In order to estimate the complexity of the whole procedure (and hence understand its computational feasibility), we need to know how many mTables exist for fix $k$ and $p$.
%
%\note[ChaTo]{@Tom: please check if the following sentence that I added is correct, I thought it would make this more understandable.}
This is the number of non-decreasing sequences of integers that end with a number smaller or equal to $k$ and have length $k$.
%
\begin{theorem}
	\label{theorem:05:number-of-mtables}
	The number of legal mTables for $k,p$ is less or equal to $\frac{k(k-1)}{2}$ .
\end{theorem}

\begin{proof}
	\label{proof:05:number-of-mtables}
	Given the proof of Theorem~\ref{theorem:05:mtable-mass-injection} we can count the number of legal mTables for fix $p,k$ as follows: The maximum mass of a legal mTable of length $k$ is by construction
	$L_1 ((m(1) = 1,m(2) = 2, \ldots, m(k) = k)) = \sum_{i=1}^k m(i) = \frac{k(k-1)}{2}$. 
	%
	Following definition~\ref{def:05:valid-mtable} the 	    entry $m(1)$ is the smallest entry or equal to all other entries. 
	%
	Furthermore, because this mTable is legal and following Definition~\ref{def:05:legal-mtable}, the mTable is a result of Algorithm~\ref{alg:constructMTable}. 
	%
	Thus $m(1) = F^{-1}(1,p,\alpha) \in \lbrace 0,1\rbrace$. In other words, $m(1)$ can only be $0$ or $1$.
	
	%
	In turn $m(2)$ can only be $2$, if $m(1)$ was $1$ (otherwise $m(2)<2$).
	%
	Accordingly, the minimum mass of a legal mTable is $L_1((m(1),m(2), \ldots, m(k))) = 0$, if all $m(i)=0$.
	%
	Following Lemma~\ref{lemma:05:non-decreasing-with-alpha-mtable}, we can create mTables with higher masses by increasing $\alpha$. 
	%
	Furthermore, following Theorem~\ref{theorem:05:mtable-mass-injection}, if a legal mTable exists for a given mass and parameters $k,p$, then this is the only existing legal mTable with that mass.
	%
	We know that there is a theoretical minimum mass for legal mTables ($L_1 =0$) which would occur, for example, if we set $\alpha = 0$ assuming $p<1$.
	%
	There also exists a theoretical maximum mass for legal mTables which is $\frac{k(k-1)}{2}$. 
	%
	At best, we can achieve every possible mass between those two to extremes. 
	%
	It follows that there are at most $\frac{k(k-1)}{2}$ masses for legal mTables of size $k$ for a fix~$p$.
\end{proof}

\subsubsection{\algoMtable complexity}\label{subsubsec:construct-mtable-complexity}
\algoMtable computes the inverse binomial cdf for k positions of the ranking and stores each of the computed values.
%
This leads to a time complexity of $\mathcal{O}(k) \cdot \mathcal{O}(inverseBinomialCDF(p,k,\alpha))$.  Assuming a constant time for the calculation of the binomial probability mass function, the time complexity of the implementation for the inverse binomial cdf which we used is in $\mathcal{O}(i^2)$ where $i$ is the position we calculate the inverse binomial cdf for.
%
Note that the complexity of the inverse binomial cdf is dependent on how accurate the computation is.
%
The space complexity is $\mathcal{O}(k)$ if we do not store any intermediate results for future calculations.
%
\subsubsection{\algoRecursive complexity}\label{subsubsec:success-prob-complexity}
The algorithm \algoRecursive has time complexity $\mathcal{O}(\prod_{j=1}^{m(k)}b(j) \cdot O(\texttt{binomPDF}))$ as explained in section \ref{subsubsec:adjustment-binomial}.
%
Before that we have to calculate the mTable and blocks $b$ which adds $\mathcal{O}(k) \cdot \mathcal{O}(F^{-1}(p,k,\alpha))$ and $\mathcal{O}(k)$ respectively to the complexity.
%
Overall we get $\mathcal{O}(k) \cdot \mathcal{O}(F^{-1}(p,k,\alpha)) + \mathcal{O}(\prod_{j=1}^{m(k)}b(j) \cdot O(\texttt{binomPDF}))$.
%
For the sake of readability we will write $\mathcal{O}($\algoMtable$) + \mathcal{O}(\prod_{j=1}^{m(k)}b(j) \cdot O(\texttt{binomPDF}))$.
%
The space complexity is $\mathcal{O}(k)$ for the maximum number of blocks plus $\mathcal{O}(k)$ for the stored probabilities at each position.
%
\subsubsection{\algoBinomBinary complexity}\label{subsubsec:binom-binary-complexity}
The binary search for $\alpha_c$ has a complexity of $\mathcal{O}(\log{}\frac{k(k-1)}{2}) = \mathcal{O}(\log{}k^2)$, because maximum $\frac{k(k-1)}{2}$ different valid mTables exist.
%
For each binary search step we need $\mathcal{O}(\mathcal{O}(k) \cdot \mathcal{O}(F^{-1}(p,k,\alpha)))$ to compute the new mTable as well as $\mathcal{O}(\prod_{j=1}^{m(k)}b(j) \cdot O(\texttt{binomPDF}))$ for its fail probability.
%
Overall we get $\mathcal{O}(\log{}k^2) \cdot (\mathcal{O}(\prod_{j=1}^{m(k)}b(j) \cdot O(\texttt{binomPDF})) + \mathcal{O}(\mathcal{O}(k) \cdot \mathcal{O}(F^{-1}(p,k,\alpha))))$, which we will write as $\mathcal{O}(\log{}k^2) \cdot (\mathcal{O}(\texttt{\algoRecursive}))$.
%
The space complexity is $\mathcal{O}(k)$ since we only store the three MTables with their respective fail probability.
\section{Conclusions}\label{sec:conclusions}

In this paper we presented the extension of \algoFAIR to multiple groups, where we guarantee ranked group fairness, without introducing a large utility loss.
%
Especially when groups largely have the same utility score in the top positions (as is the case in the COMPAS experiments) no ranking utility at all is lost in terms of NDCG or individual fairness.
%
In this case \algoFAIR only benefits the protected groups without skewing the ranking result.
%
If a protected group already receives advantageous exposure in the colorblind ranking and the ranked group fairness condition is already met via the ranking scores of the candidates, \algoFAIR preserves this.
%
A protected candidate can only lose exposure due to a protected candidate from another group being ranked up, but not due to a non-protected one.
%
Additionally the user can control the degree of fairness that is obtained in the result by setting $p_G$ to a value that is appropriate for the situation at hand.
%
This lets them transparently control the trade-off between fairness and utility, instead of having a less intuitive fairness parameter $\theta$ that operates on the barycenter of group distributions.

\spara{Future work.}
%
An important challenge is the algorithmic complexity of calculating the mTree for a particular configuration of $p_G$ and $\alpha$.
%
Though we already implemented improvements to reduce complexity, calculating an adjusted mTree of length $k=100$ with six groups takes several weeks.
%
Of course this mTree has to be calculated only once and can then be persistent and shared among users of \algoFAIR, however when a situation demands a new configuration these calculation times are currently unavoidable.
%
A significant speed-up could be achieved by programming a customized mcdf-function which can store the results of repetitive computation steps.
%
This is however very memory-intensive and the algorithm then needs to run on large computer clusters.
%
Additionally one could provide a script that fills the MCDF Cache with various configurations of $p_G$ and $\alpha$.
%
This calculation can continuously run as a separate process on a server which then provides the obtained caches to users who want to compute new mTrees.

One of the main challenges for fair ranking algorithms in general is that there is not yet much empirical evidence that re-ordering items actually helps to overcome the bias in click-probability across groups.
%
Very recent research however \cite{suhr2020does} suggests that guarantees for a minimum representation of underrepresented groups yield to higher selection rates in different hiring contexts, but does not mitigate user biases completely. For example if a user prefers male candidates for a moving assistance task over female candidates, ranking female candidates higher will not mitigate users' biases completely.
%
Thus, a method such as \algoFAIR may be able to increase the click probability for protected groups. Furthermore, setting the values for $p_G$ higher than desired may mitigate user biases. For example, it might be effective to set the minimum proportion for the protected group women in the context of hiring for moving assistance to $60\%$ in order to achieve a click probability of $50\%$ for this group.
%
However further research has to be conducted to study the effect on users of re-ordering items in a ranking and to understand the best means to overcome these strong prejudices against minority groups in certain domains.
%
Additionally, further experimental research using synthetic data could allow us to test with a wider range of differences across groups, larger than the one that real datasets exhibit.
%
Finally, robustness tests to measure the sensitivity of the rankings to noise in the qualification/score inputs could be helpful to determine to what extent they may affect our fairness objectives.

\spara{Reproducibility.}
Code and data that can be used to reproduce the experiments on this paper is available: \url{https://github.com/MilkaLichtblau/Multinomial_FA-IR}.

\section{Acknowledgements}

This research was supported by the Max Planck Institute for Software Systems, the German Research Foundation and the Catalonia Trade
and Investment Agency (ACCI{\'O}). M.Z. was supported by the MPI and the GRF. %T.S. was supported by his parents.


%%%%%%%%%%%%%%%%%%%%%%%%%%%%%%%%%%%%%%%%%%%%%%%%%%%%%%%%%%%%%%%%%%
% BIBLIOGRAPHY
%%%%%%%%%%%%%%%%%%%%%%%%%%%%%%%%%%%%%%%%%%%%%%%%%%%%%%%%%%%%%%%%%%

% This must be close to the FIRST column of the last page
% Otherwise it's too late to balance
%\balance

% Bibliography style and items
\bibliographystyle{ACM-Reference-Format}
\bibliography{main}
\appendix
\input{10-appendix}
\end{document}
\endinput
%%
