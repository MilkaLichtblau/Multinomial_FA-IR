%!TEX root = main.tex

\begin{figure}[h]
	\centering
	\begin{forest}
		for tree={
			child anchor=west,
			parent anchor=east,
			grow'=east,
			draw,
			anchor=west,
		}
		[{[0, 0]}
		[{[0, 0]}
			[{[1, 0]}
				[{[2, 0]$^*$}
					[{[3, 0]}
						[{[3, 1]}
							[{[3, 1]}, name=doubled5
								[{[4, 1]}
									[{[5, 1]}]
									[{[4, 2]}]
								]
								[{[3, 2]}, name=parentDoubled8]
							]
						]
					]
					[{[2, 1]}, name=parentDoubled2]
				]
				[{[1, 1]}, name=doubled1
					[{[1, 1]}
						[{[2, 1]}, name=doubled2
							[{[2, 2]}, name=doubled6, before drawing tree={y-=1em}
								[{[2, 2]}, before drawing tree={y-=1em}
									[{[3, 2]}, name=doubled8, before drawing tree={y-=1em}]
									[{[2, 3]}, name=doubled9, before drawing tree={y-=1em}]
								]
							]
						]
						[{[1, 2]}, name=doubled3]
					]
				]
			]
			[{[0, 1]}, name=parentDoubled1
				[{[0, 2]}
					[{[1, 2]}, name=parentDoubled3]
					[{[0, 3]}
						[{[1, 3]}
							[{[1, 3]}, name=doubled7
								[{[2, 3]}, name=parentDoubled9]
								[{[1, 4]}
									[{[2, 4]}]
									[{[1, 5]}]
								]
							]
						]
					]
				]
			]
		]]
		\draw (parentDoubled1.east)--(doubled1.west);
		\draw (parentDoubled2.east)--(doubled2.west);
		\draw (parentDoubled3.east)--(doubled3.west);
		\draw (doubled2.east)--(doubled5.west);
		\draw (doubled3.east)--(doubled6.west);
		\draw (doubled3.east)--(doubled7.west);
		\draw (parentDoubled8.east)--(doubled8.west);
		\draw (parentDoubled9.east)--(doubled9.west);
	\end{forest}
	\caption{Example of an mTree with two protected groups with minimum proportions $ p_G=\langle 1/3, 1/3 \rangle $ and $ \alpha=0.1 $. The notation $[x,y]$ indicates that in that node we have $x$ elements of group $1$ and $y$ elements of group $2$; group 0 which is the non-protected group is always unconstrained.
	%
	Levels go from left to right starting from 0. For instance, the node marked ``[0,2]$^*$'' indicates that when 3 elements are present (level 3), one of the acceptable configurations is to have 2 or more elements from group 1 and 0 elements from group 2.
	%
	We see that in case of multiple protected groups, there are various ways of satisfying definition~\ref{def:ranked-group-fairness-condition}.
	%
	Constructivly, each path in the tree corresponds to one valid strategy to place protected candidates in the ranking.
	\label{fig:mtree-symmetric-unadjusted}}
\end{figure}
%
\begin{figure}[h]
	\centering
	\begin{forest}
		for tree={
			child anchor=west,
			parent anchor=east,
			grow'=east,
			draw,
			anchor=west,
		}
		[{[0, 0]}
		[{[0, 0]}
			[{[1, 0]}
				[{[2, 0]}, before drawing tree={y-=1em}
					[{[2, 1]}, before drawing tree={y-=1em}
						[{[2, 1]}, before drawing tree={y-=1em}
							[{[2, 1]}, name=doubled5, before drawing tree={y-=2em}
								[{[2, 2]},  before drawing tree={y-=2em}
									[{[2, 2]}, name=doubled6, before drawing tree={y-=2em}
										[{[3, 2]},  before drawing tree={y-=2em}]
									]
								]
							]
						]]]
				[{[1, 1]}
				[{[1, 1]}, name=doubled1
					[{[1, 1]}, name=parentDoubled5
						[{[1, 2]}, name=doubled2, before drawing tree={y-=1em}
							[{[1, 2]}, name=parentDoubled6, before drawing tree={y-=1em}
								[{[1, 3]}, name=doubled3, before drawing tree={y-=2em}
									[{[2, 3]},  before drawing tree={y-=2em}]
									[{[1, 4]}, name=doubled4, before drawing tree={y-=2em}]
								]
							]
						]
					]]]
			]
			[{[0, 1]}
				[{[0, 1]}, name=parentDoubled1
					[{[0, 2]}
						[{[1, 2]}, name=parentDoubled2]
						[{[0, 3]}
							[{[0, 3]}
								[{[1, 3]}, name=parentDoubled3]
								[{[0, 4]}
									[{[1, 4]}, name=parentDoubled4]
									[{[0, 5]}
										[{[1, 5]}]
									]
								]
							]
						]
					]
				]]
		]]
		\draw (parentDoubled1.east)--(doubled1.west);
		\draw (parentDoubled2.east)--(doubled2.west);				
		\draw (parentDoubled3.east)--(doubled3.west);
		\draw (parentDoubled4.east)--(doubled4.west);				
		\draw (parentDoubled5.east)--(doubled5.west);
		\draw (parentDoubled6.east)--(doubled6.west);
	\end{forest}
	\caption{Example of an mTree with two protected groups with minimum proportions $ p_G=\langle 0.2, 0.4 \rangle $ and $ \alpha=0.1 $. We see that the tree, in contrast to the mTree in Figure~\ref{fig:mtree-symmetric-unadjusted}, is not symmetric because the minimum proportions $ p_G $ differ.
	\label{fig:mtree-asymmetric-unadjusted}}
\end{figure}
