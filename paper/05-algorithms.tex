\section{Algorithm}\label{sec:algorithms}
We present the multinomial \algoFAIR algorithm (\S\ref{subsec:algorithm-description}) and prove it is correct (\S\ref{subsec:algorithm-correctness}).

\subsection{Algorithm Description}\label{subsec:algorithm-description}

Multinomial \algoFAIR, presented in Algorithm~\ref{alg:fair}, solves the {\sc Fair Top-$k$ Ranking} problem for multinomial protected groups and intersectional group settings..
%
As input, multinomial \algoFAIR takes 
the expected size $k$ of the ranking to be returned,
the qualifications $q_c$, 
indicator variables $g_c$ indicating if candidate $c$ is protected,
the vector of minimum target proportions $p$ of protected candidates, and
the adjusted significance level $\alphaadj$.

First, the algorithm uses $q_c$ to create priority queues with up to $k$ candidates each: $P_0$ for the non-protected candidates and $P_g$ for the protected candidates of group $g$.
%
Next (lines \ref{alg:fair:mstart}-\ref{alg:fair:mend}), the algorithm derives a ranked group fairness tree (mTree) similar to Figure~\ref{fig:mtree-asymmetric-adjusted}, i.e., for each position it computes the minimum number of protected candidates per group, given $p_G$, $k$ and $\alphaadj$.
%
Then, multinomial \algoFAIR greedily constructs a ranking subject to candidate qualifications, and minimum protected elements required.
%
If the previously computed mTree $MT$ demands a protected candidate from group $g$ at the current position, the algorithm appends the best candidate from $P_g$ to the ranking (Lines \ref{alg:fair:pstart}-\ref{alg:fair:pend}); otherwise, it appends the best candidate from $P_0 \cup P_1 \cup \ldots \cup P_|G|$ (Lines \ref{alg:fair:anystart}-\ref{alg:fair:anyend}).
%
In case $MT$ branches into different possibilities to satisfy ranked group fairness (see Fig.~\ref{fig:mtree-asymmetric-adjusted} for $k=4$), the algorithm chooses the branch that has a higher value for $F$, meaning it chooses the branch that has a higher probability.
%
If two branches are equally likely, which happens for $p_1 = p_2 = \ldots = p_{|G|}$, then one of them is chosen at random.

\begin{algorithm}[h]
	%\caption{Algorithm \algoFAIR, finding a ranking that maximizes utility subject to in-group monotonicity and ranked group fairness constraints.}
	\caption{Algorithm \algoFAIR finds a ranking that maximizes utility subject to in-group monotonicity and ranked group fairness constraints. Checks for special cases (e.g., insufficient candidates of a class) are not included for clarity.}
	\label{alg:fair}  % But whenever possible refer to this algo. by name not number
	\small
	\AlgInput{$k \in [n]$, the size of the list to return; $\forall~c \in [n]$: $q_c$, the qualifications for candidate $c$, and $g_c$ an indicator that is $>0$ iff candidate $c$ is protected; $p_G$ with $\forall p \in p_G \in ]0,1[$, the vector of minimum proportions for each group of protected elements; $\alphaadj \in ]0,1[$, the adjusted significance for each fair representation test.}
	\AlgOutput{$\tau$ satisfying the group fairness condition with parameters $p, \sigma$, and maximizing utility.}
	%\AlgComment{compute min. protected candidates per position}
	$P_0, P_1, \ldots P_{|G|} \leftarrow$ empty priority queues with bounded capacity $k$\\
	\For{$c \leftarrow 1$ \KwTo $n$}{
		insert $c$ with value $q_c$ in priority queue $P_{g_c}$ \\
	}
	\For{$i \leftarrow 1$ \KwTo $k$}{ \label{alg:fair:mstart}
		$m(i) \leftarrow F^{-1}(\alphaadj; i, p)$ \\
	}\label{alg:fair:mend}
	%\AlgComment{create fair ranking}
	$(t_p, t_n) \leftarrow (0, 0)$ \\
	\While{$t_p + t_n < k$}{
		\eIf{$t_p < m[t_p + t_n + 1]$}{
			\AlgComment{add a protected candidate}
			$t_p \leftarrow t_p + 1$ \\ \label{alg:fair:pstart}
			$\tau[t_p + t_n] \leftarrow \operatorname{dequeue}(P_1)$ \\  \label{alg:fair:pend}
		}{
			\AlgComment{add the best candidate available}
			\eIf{$q(\operatorname{peek}(P_1)) \ge q(\operatorname{peek}(P_0))$} { \label{alg:fair:anystart}
				%\AlgComment{best one is a protected candidate}
				$t_p \leftarrow t_p + 1$ \\
				$\tau[t_p + t_n] \leftarrow \operatorname{dequeue}(P_1)$ \\
				
			}{
				%\AlgComment{best one is a non-protected candidate}
				$t_n \leftarrow t_n + 1$ \\
				$\tau[t_p + t_n] \leftarrow \operatorname{dequeue}(P_0)$ \\
			} \label{alg:fair:anyend}
		}
		
	}
	\Return{$\tau$}
\end{algorithm}
\vspace{-3mm}

\subsection{Algorithm Complexity}
\todo{Not done yet}

\begin{table}[]
\caption{Space and time complexity for all algorithms.
		\label{tbl:space_time}}
\begin{tabular}{|c|c|c|}
\hline
\textbf{Algorithm} & \textbf{Time Complexity} & \textbf{Space Complexity} \\ \hline
inverseBinomialCDF & $\mathcal{O}(n\log{}n)$ & $\mathcal{O}(n\log{}n)$ \\ \hline
\algoMtable & $\mathcal{O}(n\log{}n)$ & $\mathcal{O}(n\log{}n)$ \\ \hline
\algoRecursive & $\mathcal{O}(n\log{}n)$ & $\mathcal{O}(n\log{}n)$ \\ \hline
\algoBinomBinary & $\mathcal{O}(n\log{}n)$ & $\mathcal{O}(n\log{}n)$ \\ \hline
multinomialCDF & $\mathcal{O}(n\log{}n)$ & $\mathcal{O}(n\log{}n)$ \\ \hline
\algoImcdf & $\mathcal{O}(n\log{}n)$ & $\mathcal{O}(n\log{}n)$ \\ \hline
\algoComputeMTree & $\mathcal{O}(n\log{}n)$ & $\mathcal{O}(n\log{}n)$ \\ \hline
\algoMultBinary & $\mathcal{O}(n\log{}n)$ & $\mathcal{O}(n\log{}n)$ \\ \hline
\algoReg & $\mathcal{O}(n\log{}n)$ & $\mathcal{O}(n\log{}n)$ \\ \hline
\algoFAIR & $\mathcal{O}(n\log{}n)$ & $\mathcal{O}(n\log{}n)$ \\ \hline
\end{tabular}
\end{table}

\algoFAIR has running time $O(n + k \log k)$; which includes building the $O(k)$ size priority queues from $n$ items and processing them to obtain the final ranking, where we assume $k < O(n/\log n)$. 
\meike{Was soll denn $k < O(n/\log n)$ heißen? Warum ist die size der priority queues O(k) und nicht k?}
%
If we already have ranked lists for all groups of elements, \algoFAIR can avoid the first step and obtain the top-$k$ in $O(k \log k)$ time.
%
Our method is applicable as long as there is at least one protected group and there are enough candidates in each protected group; if there are $k$ from each group, the algorithm is guaranteed to succeed, otherwise the ``head'' of the ranking will satisfy the ranked group fairness constraint, but the ``tail'' of the ranking may not.

\subsection{Algorithm Optimizations}
Because the computation of an adjusted mTree is expensive (Table~\ref{tbl:space_time}), our implementation persists already computed mTrees and their components to never do the same computation twice. 
%
Depending on the structure of $p_G$, different levels of optimization are applicable.

We note however that an mTree has to be computed only once for a particular combination of $k, p_G, \alpha$ and that \algoFAIR has a complexity of $O(k log k)$.
%
We provide the already pre-computed mTrees and MCDF caches for our experiments and the intermediate steps not only for reproducibility but for use in practice too.

\subsubsection{MCDF Cache}\label{subsubsec:mcdf-cache}
Table~\ref{tbl:space_time} shows that the highest computational cost arises from computing the multinomial cumulative distribution function $F$. 
%
In the worst case Algorithm~\ref{alg:imcdf} computes it $|G|+1$ times for each group $g$ in $m_g(i)$ and each position $i\leq k$.
%
However, the same calculation may be done many times:
%
As an example consider the (fictive) mTree nodes $[2,1]$ and $[1,2]$ at position $k=3$. 
%
To compute the successors of node $[2,1]$ we call Algorithm~\ref{alg:imcdf} with arguments $(4,[2,1])$, $(4,[3,1])$ and $(4,[2,2])$. 
%
We store the results of these calculation in a map that we call MCDF cache with the algorithm arguments ($k$ and the minimum protected candidates of each group) as key and the corresponding mcdf as value.
%
Next we compute the successors of node $[1,2]$ and call Algorithm~\ref{alg:imcdf} with arguments $(4,[1,2])$, $(4,[2,2])$ and $(4,[1,3])$. 
%
We see that we would compute the mcdf for $(4,[2,2])$ twice, but instead we can now read it from the MCDF cache.

Furthermore, if our example has symmetric minimum proportions $p_1 = p_1$, the mcdf of $(4,[2,1])$ is equal to $(4,[1,2])$ and $\textit{mcdf}(4,[1,3]) = \textit{mcdf}(4,[3,1])$. 
%
Generally, if $p_1 = p_2 = \cdots = p_{|G|}$, we can make use of the mTree's symmetry: we calculate the mcdf only for node $m(i)$ and store it in the cache \emph{as well as its mirror} (see Section~\ref{subsubsec:discarding-symmetric-nodes}), because their mcdf values are the same.

Note that the mcdf computation is only depends on $p_G$ and not on $alpha$. 
%
We can therefore persists the MCDF cache on disk for a particular vector $p_G$ and load it for any mTree calculation with the same $p_G$ in the future.
%
This also saves additional computation time during significance adjustment.

\subsubsection{Discarding symmetric nodes}
\label{subsubsec:discarding-symmetric-nodes}
In case of equal minimum proportions for all groups $p_1 = p_2 = \ldots = p_|G|$ the mTree shows a convenient property that we can use to reduce additional space and computation time. 
%
Remember that whenever the mcdf value falls below $\alpha$ for a particular position $i$, we have to put a protected candidate onto $i$. 
%
For equal minimum proportions the tree branches into $|G| - 1$ symmetric nodes $m(i)$ of the same likelyhood.
%
As an example reconsider the mTree from Figure~\ref{fig:mtree-symmetric-adjusted} at level 6. 
%
For two protected groups with minimum proportions $[1/3, 1/3]$ we see that the tree branches into two symmetric nodes $[1, 0]$ and $[0,1]$. 
%
Both have the same mcdf values.
%
We store only one of the nodes and flag it as ``has mirrored node'' and continue our mTree computation only in the stored branch.
%
This way we save half of the space and computation time needed, without loosing any information about the tree. 

\subsubsection{Stored mTrees}
\label{subsubsec:stored-mtrees}
During the computation of an adjusted mTree with parameters $k,p_G , \alpha$ we calculate many temporary mTrees (first the unadjusted ones, then the ones for the regression algorithm, then the ones for the binary search steps).
%
We persist all of the temporary mTrees plus the final tree in files for later usage.
%
The filenames contain the tree parameters and whether or not it is adjusted and its probability to fail a fair ranking $\failprob$.
%
If any of these trees is needed at a later point in time it can be loaded from disc instead of being recomputed, be it as input for multinomial \algoFAIR or as temporary tree during a new adjusted mTree computation.

\subsection{Algorithm Correctness}\label{subsec:algorithm-correctness}
By construction, a ranking $\tau$ generated by \algoFAIR satisfies in-group monotonicity, because all candidates are selected by decreasing qualifications.
%
It also satisfies the ranked group fairness constraint, because for every prefix of size $i$ the list, the number of protected candidates in group $g$ is at least $m_g(i)$. 
%
What we must prove is that $\tau$ achieves optimal selection utility, and that it maximizes ordering utility. 
%
This is done in the following lemmas.

\begin{lemma}\label{lemma:across}
	If a ranking satisfies the in-group monotonicity constraint, then the utility loss (ordering or selection utility different from zero) can only happen across protected/non-protected groups.
\end{lemma}

\begin{proof}
	This comes directly from Definition~\ref{def:inGroupMonotonicity} given that for two candidates $c,d$, the only case in which $r(c,\tau) < r(d,\tau) \wedge q_c < q_d$ is when $g_c \ne g_d$.
\end{proof}

\begin{lemma}
	The optimal selection utility among rankings satisfying in-group monotonicity (\ref{problem:constraint-monotonicity}) and ranked group fairness (\ref{problem:constraint-rank}), is either zero, or is due to a non-protected candidate ranked below a less qualified protected candidate.
\end{lemma}

\begin{proof}
	Let $c,d$ be the two candidates that attain the optimal selection utility, with $c \in \tau, d \in [n] \backslash \tau$.
	%
	We will prove this by contradiction: let us assume $c$ is a non-protected candidate ($g_c=0$) and $d$ is a protected element ($g_d>0$). 
	%
	Let us swap $c$ and $d$, moving $c$ outside $\tau$ and $d$ inside the ranking, and then moving down $d$ if necessary to place it in the correct ordering among the protected elements below its position (given that $c$ is the last non-protected element in $\tau$). 
	%
	The new ranking continues to satisfy in-group monotonicity as well as ranked group fairness (as it has not decreased the number of protected elements at any position in the ranking), and has a larger selection utility. 
	%
	This is a contradiction because the selection utility was optimal. Hence, $c$ is a protected element and $d$ a non-protected element.
\end{proof}

\meike{todo from here}
\begin{lemma}\label{lemma:number-protected-implies-selfairness} %[Elements inside/outside a ranking are determined by $\tau_p$]
	Given two rankings $\rho, \tau$ satisfying in-group monotonicity (\ref{problem:constraint-monotonicity}), if they have the same number of protected elements $\rho_p = \tau_p$, then both rankings contain the same $k$ elements (possibly in different order), and hence both rankings have the same selection utility.
\end{lemma}

\begin{proof}
	Both rankings contain a prefix of size $\tau_p$ of the list of protected candidates ordered by decreasing qualifications, and a prefix of size $k - \tau_p$ of the list of non-protected candidates ordered by decreasing qualifications. 
	%
	Hence, $\forall i \in [n], i \in \tau \Leftrightarrow i \in \rho$, so the elements not included in the rankings are also the same elements, and the selection utility of both rankings is the same.
\end{proof}

The previous lemma means selection utility is determined by the number of protected candidates in a ranking.

\begin{lemma}\label{lemma:fair-optimal-selection}
	Algorithm \algoFAIR achieves optimal selection utility among rankings satisfying in-group monotonicity (\ref{problem:constraint-monotonicity}) and ranked group fairness (\ref{problem:constraint-rank}).
\end{lemma}

\begin{proof}
	Let $\tau$ be the ranking produced by \algoFAIR, and $\tau^*$ be the ranking achieving the optimal selection utility. We will prove that $\tau_p = \tau^*_p$ by contradiction.
	%
	Suppose $\tau_p < \tau^*_p$. Then, we could take the least qualified protected element in $\tau^*_p$ and swap it with the most qualified non-protected element in $[n] \backslash \tau^*_p$, re-ordering as needed. This would increase selection utility and still satisfy the constraints, which is a contradiction with the fact that $\tau^*_p$ achieved the optimal selection utility.
	%
	Suppose $\tau_p > \tau^*_p$. Then, at the position at which the least qualified protected element in $\tau$ is found, we could have placed a non-protected element with higher qualifications, as $\tau^*$ satisfies ranked group fairness and has less protected elements. This is a contradiction with the way in which \algoFAIR operates, as it only places a protected element with lower qualifications when needed to satisfy ranked group fairness.
	%
	Hence, $\tau_p = \tau^*_p$ and by Lemma~\ref{lemma:number-protected-implies-selfairness} it achieves the same selection utility.
\end{proof}

\begin{lemma}
	Algorithm \algoFAIR maximizes ordering utility among rankings satisfying in-group monotonicity (\ref{problem:constraint-monotonicity}), ranked group fairness (\ref{problem:constraint-rank}), and achieving optimal selection utility (\ref{problem:optimal-sel}).
\end{lemma}

\begin{proof}
	By lemmas~\ref{lemma:number-protected-implies-selfairness} and \ref{lemma:fair-optimal-selection} we know that satisfying the constraints and achieving optimal selection utility implies having a specific number of protected elements $\tau^*_p$.
	%
	Hence, we need to show that among rankings having this number of protected elements, \algoFAIR achieves the maximum ordering utility.
	%
	By Lemma~\ref{lemma:across} we know that loss of ordering utility is due only to non-protected elements placed below less qualified protected elements. 
	%
	However, we know that in \algoFAIR this only happens when necessary to satisfy ranked group fairness, and having less protected elements at any given position than the ranking produced by \algoFAIR would violate the ranked group fairness constraint.
	%However, we know that in \algoFAIR this only happens when necessary to satisfy ranked group fairness, similarly to~\cite{celis2017ranking}, and by the same arguments, having less protected elements at any given position than the ranking produced by \algoFAIR would violate the ranked group fairness constraint.
\end{proof}



