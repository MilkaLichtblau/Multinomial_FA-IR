%!TEX root = main.tex

\begin{figure}[t!]
	\centering
	\resizebox{1\columnwidth}{!}{%
	\begin{forest}
		for tree={
			child anchor=west,
			parent anchor=east,
			grow'=east,
			draw,
			anchor=west,
		}
		[{1, [0, 0]}
		[{2, [0, 0]}
			[{3, [1, 0]}
				[{4, [2, 0]$^*$}
					[{5, [3, 0]}
						[{6, [3, 1]}
							[{7, [3, 1]}, name=doubled5
								[{8, [4, 1]}
									[{9, [5, 1]}]
									[{9, [4, 2]}]
								]
								[{8, [3, 2]}, name=parentDoubled8]
							]
						]
					]
					[{5, [2, 1]}, name=parentDoubled2]
				]
				[{4, [1, 1]}, name=doubled1
					[{5, [1, 1]}
						[{6, [2, 1]}, name=doubled2
							[{7, [2, 2]}, name=doubled6, before drawing tree={y-=1em}
								[{8, [2, 2]}, before drawing tree={y-=1em}
									[{9, [3, 2]}, name=doubled8, before drawing tree={y-=1em}]
									[{9, [2, 3]}, name=doubled9, before drawing tree={y-=1em}]
								]
							]
						]
						[{6, [1, 2]}, name=doubled3]
					]
				]
			]
			[{3, [0, 1]}, name=parentDoubled1
				[{4, [0, 2]}
					[{5, [1, 2]}, name=parentDoubled3]
					[{5, [0, 3]}
						[{6, [1, 3]}
							[{7, [1, 3]}, name=doubled7
								[{8, [2, 3]}, name=parentDoubled9]
								[{8, [1, 4]}
									[{9, [2, 4]}]
									[{9, [1, 5]}]
								]
							]
						]
					]
				]
			]
		]]
		\draw (parentDoubled1.east)--(doubled1.west);
		\draw (parentDoubled2.east)--(doubled2.west);
		\draw (parentDoubled3.east)--(doubled3.west);
		\draw (doubled2.east)--(doubled5.west);
		\draw (doubled3.east)--(doubled6.west);
		\draw (doubled3.east)--(doubled7.west);
		\draw (parentDoubled8.east)--(doubled8.west);
		\draw (parentDoubled9.east)--(doubled9.west);
	\end{forest}
	}
	\CaptionMargin
	\caption{Example of an mTree with two protected groups with minimum proportions $ p_G=\langle 1/3, 1/3 \rangle $ and $ \alpha=0.1 $. \changed{The notation $(k, [x,y])$ indicates that at ranking position $k$, we need at least $x$ elements of group $1$ and $y$ elements of group $2$ to satisfy ranked group fairness; group 0, which is the non-protected group, is always unconstrained.
	%
	%Levels go from left to right starting from 1, as indicated by the first number in the nodes.
	For instance, the node marked ``(4, [2, 0])$^*$'' indicates that in the first 4 positions of this ranking, one of the acceptable configurations is to have 2 or more elements from group 1 and 0 or more elements from group 2.}
	%
	We see that in case of multiple protected groups, there are various ways of satisfying Definition~\ref{def:ranked-group-fairness-condition}.
	%
	Thus, each path in the tree corresponds to one valid strategy to place protected candidates in the ranking.
	\label{fig:mtree-symmetric-unadjusted}}
\end{figure}
%
%\begin{figure}[t!]
%	\centering
%	\begin{forest}
%		for tree={
%			child anchor=west,
%			parent anchor=east,
%			grow'=east,
%			draw,
%			anchor=west,
%		}
%		[{[0, 0]}
%		[{[0, 0]}
%			[{[1, 0]}
%				[{[2, 0]}, before drawing tree={y-=1em}
%					[{[2, 1]}, before drawing tree={y-=1em}
%						[{[2, 1]}, before drawing tree={y-=1em}
%							[{[2, 1]}, name=doubled5, before drawing tree={y-=2em}
%								[{[2, 2]},  before drawing tree={y-=2em}
%									[{[2, 2]}, name=doubled6, before drawing tree={y-=2em}
%										[{[3, 2]},  before drawing tree={y-=2em}]
%									]
%								]
%							]
%						]]]
%				[{[1, 1]}
%				[{[1, 1]}, name=doubled1
%					[{[1, 1]}, name=parentDoubled5
%						[{[1, 2]}, name=doubled2, before drawing tree={y-=1em}
%							[{[1, 2]}, name=parentDoubled6, before drawing tree={y-=1em}
%								[{[1, 3]}, name=doubled3, before drawing tree={y-=2em}
%									[{[2, 3]},  before drawing tree={y-=2em}]
%									[{[1, 4]}, name=doubled4, before drawing tree={y-=2em}]
%								]
%							]
%						]
%					]]]
%			]
%			[{[0, 1]}
%				[{[0, 1]}, name=parentDoubled1
%					[{[0, 2]}
%						[{[1, 2]}, name=parentDoubled2]
%						[{[0, 3]}
%							[{[0, 3]}
%								[{[1, 3]}, name=parentDoubled3]
%								[{[0, 4]}
%									[{[1, 4]}, name=parentDoubled4]
%									[{[0, 5]}
%										[{[1, 5]}]
%									]
%								]
%							]
%						]
%					]
%				]]
%		]]
%		\draw (parentDoubled1.east)--(doubled1.west);
%		\draw (parentDoubled2.east)--(doubled2.west);
%		\draw (parentDoubled3.east)--(doubled3.west);
%		\draw (parentDoubled4.east)--(doubled4.west);
%		\draw (parentDoubled5.east)--(doubled5.west);
%		\draw (parentDoubled6.east)--(doubled6.west);
%	\end{forest}
%	\CaptionMargin
%	\caption{Example of an mTree with two protected groups with minimum proportions $ p_G=\langle 0.2, 0.4 \rangle $ and $ \alpha=0.1 $. We see that the tree, in contrast to the mTree in Figure~\ref{fig:mtree-symmetric-unadjusted}, is not symmetric because the minimum proportions $ p_G $ differ.
%	\label{fig:mtree-asymmetric-unadjusted}}
%\end{figure}
