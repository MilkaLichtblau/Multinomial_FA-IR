%!TEX root = main.tex
\begin{figure}[t!]
	\centering
	\resizebox{1\columnwidth}{!}{%
	\begin{forest}
		for tree={
			child anchor=west,
			parent anchor=east,
			grow'=east,
			draw,
			anchor=west,
		}
		[{1, [0, 0]}
		[{2, [0, 0]}
		[{3, [0, 0]}
		[{4, [0, 0]}
		[{5, [0, 0]}
			[{6, [1, 0]}
			[{6, [1, 0]}
				[{7, [2, 0]}
				[{8, [2, 0]}
					[{9, [3, 0]}]
					[{9, [2, 1]}]
				]]
				[{7, [1, 1]}, name=doubled1
				[{8, [1, 1]}
				[{9, [1, 1]}
				]]]
			]]
			[{6, [0, 1]}
			[{6, [0, 1]}, name=parentDoubled1
				[{7, [0, 2]}
				[{8, [0, 2]}
					[{9, [1, 2]}]
					[{9, [0, 3]}]
				]]
			]]
		]]]]]]
		\draw (parentDoubled1.east)--(doubled1.west);
	\end{forest}
	}
	\CaptionMargin
	\caption{Example of an mTree with two protected groups with minimum proportions $ p_G=[1/3, 1/3] $ and $ \alphaadj=0.1 $. Compared to figure~\ref{fig:mtree-symmetric-unadjusted} this tree is less strict such that its \emph{total} probability $ \alphaadj $ of rejecting a fair ranking (i.e. a type-1-error) is 0.1.
	\label{fig:mtree-symmetric-adjusted}}
\end{figure}

%\begin{figure}[h]
%	\centering
%	\begin{forest}
%		for tree={
%			child anchor=west,
%			parent anchor=east,
%			grow'=east,
%			draw,
%			anchor=west,
%		}
%		[{[0, 0]}
%			[{[0, 0]}
%				[{[0, 0]}
%					[{[1, 0]}
%						[{[1, 0]}
%							[{[2, 0]}
%								[{[2, 1]}
%									[{[2, 1]}, name=doubled2, before drawing tree={y-=1em}
%										[{[2, 1]}, before drawing tree={y-=1em}
%											[{[3, 1]}, before drawing tree={y-=1em}]
%											[{[2, 2]}, before drawing tree={y-=1em}]
%										]
%									]
%								]
%							]
%							[{[1, 1]}, name=doubled1
%								[{[1, 1]}, name=parentDoubled2
%									[{[1, 2]}, name=doubled3, before drawing tree={y-=1em}
%										[{[1, 2]}, before drawing tree={y-=1em}
%											[{[1, 2]}, before drawing tree={y-=1em}]
%										]
%									]
%								]
%							]
%						]]
%					[{[0, 1]}
%						[{[0, 1]}, name=parentDoubled1
%							[{[0, 2]}
%								[{[0, 2]}, name=parentDoubled3
%									[{[0, 3]}, before drawing tree={y-=1em}
%										[{[0, 3]}, before drawing tree={y-=1em}
%											[{[1, 3]}, before drawing tree={y-=1em}]
%											[{[0, 4]}, before drawing tree={y-=1em}]
%										]
%									]
%								]
%							]
%						]]
%					]]]
%		\draw (parentDoubled1.east)--(doubled1.west);
%		\draw (parentDoubled2.east)--(doubled2.west);
%		\draw (parentDoubled3.east)--(doubled3.west);
%	\end{forest}
%	\CaptionMargin
%	\caption{Example of an mTree with two protected groups with minimum proportions $ p_G=[0.2, 0.4] $ and a corrected $ \alpha_c=0.1 $.
%	%
%	The tree is less strict than the mTree in Figure~\ref{fig:mtree-asymmetric-unadjusted}.
%	%
%	The adjusted tree yields an overall false-negative rate of $ \alpha_c=0.1 $ when testing rankings for ranked group fairness.
%	\label{fig:mtree-asymmetric-adjusted}}
%\end{figure}
